%        File: symmetries-and-polynomials-notes.tex
%     Created: Thu May 17 10:00 AM 2018 P
% Last Change: Thu May 17 10:00 AM 2018 P
%
%%%%%%%%%%%%%%%%%%%%%
%   AMS packages    %
%%%%%%%%%%%%%%%%%%%%%
% \documentclass[12 pt, reqno]{amsart}

\documentclass[reqno, 12pt, letter]{article}
\usepackage[margin=1in]{geometry}
\usepackage{mdframed}

\usepackage{hyperref}
\usepackage{etex}
\usepackage[shortlabels]{enumitem}
\usepackage{amsmath}
\usepackage{amsxtra}
\usepackage{amscd}
\usepackage{amsthm}
\usepackage{amsfonts}
\usepackage{amssymb}
\usepackage{eucal}
\usepackage[all]{xy}
\usepackage{graphicx}
\usepackage{tikz-cd}
\usepackage{mathrsfs}
\usepackage{subfiles}
\usepackage{mathpazo}
%\usepackage{euler}
\usepackage[colorinlistoftodos, textsize=tiny]{todonotes}
\usepackage{morefloats}
\usepackage{pdfpages}
\usepackage{thm-restate}
\usepackage[utf8]{inputenc}
\usepackage{epigraph}
\usepackage{csquotes}
\usepackage{gensymb}
% \usepackage[margin=1.5in]{geometry}

\graphicspath{ {images/} }

\RequirePackage{color}
\definecolor{myred}{rgb}{0.75,0,0}
\definecolor{mygreen}{rgb}{0,0.5,0}
\definecolor{myblue}{rgb}{0,0,0.65}

\usepackage{hyperref}
\hypersetup{citecolor=blue}
\usepackage{tikz}
\usetikzlibrary{matrix,arrows,decorations.pathmorphing}

\theoremstyle{plain}
\newtheorem{theorem}{Theorem}[section]
\newtheorem{proposition}[theorem]{Proposition}
\newtheorem{lemma}[theorem]{Lemma}
\newtheorem{corollary}[theorem]{Corollary}
\theoremstyle{definition}
\newtheorem{definition}[theorem]{Definition}
\newtheorem{remark}[theorem]{Remark}
\newtheorem{example}[theorem]{Example}
\newtheorem{exercise}[theorem]{Exercise}
\newtheorem{counterexample}[theorem]{Counterexample}
\newtheorem{convention}[theorem]{Convention}
\newtheorem{question}[theorem]{Question}
\newtheorem{conjecture}[theorem]{Conjecture} 
\newtheorem{goal}[theorem]{Goal}
\newtheorem{warn}[theorem]{Warning}
\newtheorem{fact}[theorem]{Fact}
\theoremstyle{remark}
\newtheorem{notation}[theorem]{Notation}
\numberwithin{equation}{section}
  
\newcommand\nc{\newcommand}
\nc\on{\operatorname}
\nc\renc{\renewcommand}
\newcommand\se{\section}
\newcommand\ssec{\subsection}
\newcommand\sssec{\subsubsection}
\newcommand\bn{{\mathbb N}}
\newcommand\bc{{\mathbb C}}
\newcommand\br{{\mathbb R}}
\newcommand\bq{{\mathbb Q}}
\newcommand\bp{{\mathbb P}}
\newcommand\CF{{\mathcal F}}
\newcommand\bz{{\mathbb Z}}
\newcommand\ba{{\mathbb A}}
\newcommand\bg{{\mathbb G}}
\newcommand\fa{{\mathfrak a}}
\newcommand\fp{{\mathfrak p}}
\newcommand\fq{{\mathfrak q}}
\newcommand\fm{{\mathfrak m}}
\newcommand\fg{{\mathfrak g}}
\newcommand\fh{{\mathfrak h}}
\newcommand\fgl{{\mathfrak {gl}}}
\newcommand\fsu{{\mathfrak {su}}}
\newcommand\fsl{{\mathfrak {sl}}}
\newcommand\fso{{\mathfrak {so}}}
\newcommand\fo{{\mathfrak {o}}}
\newcommand\fu{{\mathfrak {u}}}
\newcommand\fsp{{\mathfrak {sp}}}
\newcommand\fgsp{{\mathfrak {gsp}}}



\newcommand\sca{\mathscr A}
\newcommand\scb{\mathscr B}
\newcommand\scc{\mathscr C}
\newcommand\scd{\mathscr D}
\newcommand\sce{\mathscr E}
\newcommand\scf{\mathscr F}
\newcommand\scg{\mathscr G}
\newcommand\sch{\mathscr H}
\newcommand\sci{\mathscr I}
\newcommand\scj{\mathscr J}
\newcommand\sck{\mathscr K}
\newcommand\scl{\mathscr L}
\newcommand\scm{\mathscr M}
\newcommand\scn{\mathscr N}
\newcommand\sco{\mathscr O}
\newcommand\scp{\mathscr P}
\newcommand\scq{\mathscr Q}
\newcommand\scs{\mathscr S}
\newcommand\sct{\mathscr T}
\newcommand\scu{\mathscr U}
\newcommand\scv{\mathscr V}
\newcommand\scw{\mathscr W}
\newcommand\scx{\mathscr X}
\newcommand\scy{\mathscr Y}
\newcommand\scz{\mathscr Z}

\newcommand \ra{\rightarrow}
\newcommand \xra{\xrightarrow}
\DeclareMathOperator\spec{\text{Spec }}
\DeclareMathOperator\proj{\text{Proj }}
\DeclareMathOperator\rspec{\textit{Spec }}
\DeclareMathOperator\rproj{\textit{Proj }}
\newcommand*{\shom}{\mathscr{H}\kern -.5pt om}
\newcommand*{\stor}{\mathscr{T}\kern -.5pt or}
\newcommand*{\sext}{\mathscr{E}\kern -.5pt xt}
\newcommand \mg{{\mathscr M_g}}

\makeatletter
\newcommand{\customlabel}[2]{\protected@write \@auxout {}{\string \newlabel {#1}{{#2}{\thepage}{#2}{#1}{}} }\hypertarget{#1}{#2}}
\newcommand\sh[1]{\sco_{#1}^{\rm{sh}}}
\DeclareMathOperator\id{id}
\DeclareMathOperator\tor{Tor}
\renewcommand\hom{Hom}
\DeclareMathOperator\coker{coker}
\DeclareMathOperator\ord{ord}
\DeclareMathOperator\hilb{Hilb}
\DeclareMathOperator\rk{rk}
\DeclareMathOperator\di{div}
\DeclareMathOperator\pic{Pic}
\DeclareMathOperator\lcm{lcm}
\DeclareMathOperator\rank{rank}
\DeclareMathOperator\codim{codim}
\DeclareMathOperator\vol{Vol}
\DeclareMathOperator\supp{Supp}
\DeclareMathOperator\spn{Span}
\DeclareMathOperator\im{im}
\DeclareMathOperator\End{End}
\DeclareMathOperator\sym{Sym}
\DeclareMathOperator\pgl{PGL}
\DeclareMathOperator\sat{Sat}
\DeclareMathOperator\blow{Bl}
\renewcommand\sp{\mathrm{Sp}}
\DeclareMathOperator\gsp{GSp}
\DeclareMathOperator\sgn{sgn}
\DeclareMathOperator\gal{gal}
\DeclareMathOperator\tr{tr}
\renewcommand\char{char}
\newcommand\bbf{{\mathbb F}}
\newcommand\bk{{\Bbbk}}
\newcommand\ul{\underline}
\newcommand\ol{\overline}
\DeclareMathOperator\pr{pr}
\DeclareMathOperator\ev{ev}
\DeclareMathOperator\maj{maj}
\DeclareMathOperator\inv{inv}
\DeclareMathOperator\isom{isom}
\DeclareMathOperator\mor{mor}
\DeclareMathOperator\aut{Aut}
\DeclareMathOperator\gl{GL}
\renewcommand\sl{\mathrm{SL}}
\DeclareMathOperator\mat{Mat}
\DeclareMathOperator\stab{Stab}
\DeclareMathOperator\so{SO}
\DeclareMathOperator\su{SU}
\renewcommand\u{\mathrm{U}}
\DeclareMathOperator\lie{Lie}
\DeclareMathOperator\ad{ad}
\DeclareMathOperator\Ad{Ad}
\renewcommand\o{{\rm{O}}}

\renewcommand{\thefootnote}{\fnsymbol{footnote}}
%\newcommand{\hint}[1]{\footnote{\raggedleft\rotatebox{180}{\tiny{{Hint:} #1\hfill}}}}
\newcommand{\hint}[1]{\footnote{{Hint:} #1\hfill}}


\setcounter{MaxMatrixCols}{20}

\def\listtodoname{List of Todos}
\def\listoftodos{\@starttoc{tdo}\listtodoname}

\usepackage{endnotes}

\let\footnote=\endnote



\title{Symmetries and Polynomials Notes}
\author{Aaron Landesman and Apurva Nakade}

\usepackage{microtype}
\begin{document}

\maketitle

\todo{Apurva: I changed the documentclass and margin. We can switch if this doesn't look good, I'm just trying it out. Hints can also go in the footnotes, I think you did this in your class last year, that way not everyone sees the hints immediately.
}
\todo{aaron: I changed the footnotes to endnotes so we could give a page handout to everyone, which i think is less tempting for people to look at if they dont want to, but feel free to change it back}

\section{The Discriminant}

\subsection{Quadratic Polynomials}
\begin{definition}[Quadratic discriminant]
	\label{definition:quadratic-discriminant}
	Let $p(x) = x^2 + bx + c$ be a polynomial with $b$ and $c$ real numbers.
	The discriminant $\Delta(p)$ is by definition $b^2 - 4c$.
\end{definition}
\begin{exercise}
	\label{exercise:}
	If the polynomial $p(x) = x^2 + bx + c$ has roots $\alpha$ and $\beta$, express $b$ and $c$
	in terms of $\alpha$ and $\beta$.
\end{exercise}
\begin{exercise}
	\label{exercise:quadratic-roots}
	What does the sign of the discriminant (i.e., whether $\Delta(p) > 0, < 0$ or $=0$) 
	tell you about the roots $\alpha$ and $\beta$?
	\footnote{{\it Hint:} If the discriminant is $>0$ show that the roots are real, if it is equal
	to $0$, show they are the same, if it is less than $0$, show they are complex numbers
which are not real.}
\end{exercise}
\begin{exercise}
	\label{exercise:quadratic-discriminant-in-roots}
	Express the discriminant of the polynomial $p(x) = x^2 + bx + c$ in terms of
	the roots $\alpha, \beta$.
\end{exercise}

\subsection{The discriminant in general}
\begin{definition}
	\label{definition:discriminant}
	For $p(x)$ a polynomial of the form $p(x) = (x-r_1)(x-r_2) \cdots (x-r_n)$,
	define the discriminant $\Delta(p) := \prod_{1 \leq i < j \leq n} (r_i - r_j)^2$.
\end{definition}
\begin{exercise}
	\label{exercise:}
	Verify that for $p(x)$ of degree $2$, the definition of the discriminant
	of a general polynomial from \autoref{definition:discriminant} agrees with
	that of a quadratic polynomial given in \autoref{definition:quadratic-discriminant}.
	\footnote{{\it Hint:} Use \autoref{exercise:quadratic-discriminant-in-roots}.}
\end{exercise}
The following exercise shows that the discriminant measures, in a precise sense, how far apart
the roots of a polynomial are.
\begin{exercise}
	\label{exercise:discriminant-vanishing}
	Show that for $p$ a polynomial, $\Delta(p) = 0$ if and only if $p$ has a repeated root.
\end{exercise}
\subsection{Cubic discriminants}
\begin{exercise}
	\label{exercise:real-root-cubic}
	Show that a cubic polynomial with real coefficients $p(x) = x^3 + ax + bx + c$
	always has a real root.
	\footnote{{\it Hint:} Graph the cubic and show it intersects the line $p(x) = 0$.}
\end{exercise}
\begin{exercise}
	\label{exercise:}
	Using \autoref{exercise:real-root-cubic}, show that a cubic polynomial either has
	$3$ real roots or $2$ complex conjugates roots (of the form $a + bi, a-bi$ for $a,b$ real numbers) and one real root. 
	\footnote{{\it Hint:} Factor out the real root, and use your understanding of quadratic polynomials.}
\end{exercise}
\begin{exercise}
	\label{exercise:discriminant-cubic-sign}
	For $p(x) = x^3 + ax^2 + bx + c$ a cubic polynomial with real coefficients, show $\Delta(p) = 0$ if and only 
	if there is a repeated root, $\Delta(p) > 0$ if and only if $p$ has three distinct real roots, and $\Delta(p) < 0$ if and only if $p$ has two complex conjugate roots and one real root.
	Compare your answer to \autoref{exercise:quadratic-roots}.
\end{exercise}

\todo{aaron: i didn't discuss the formula for the cubic discriminant, but i think this is already perhaps too much material for one day, i think maybe you can just state it as a fact the formula that the discriminant of $x^3 + ax + b$ is $-4a^3-27b^2$ or whatever it is (and optionally allow them to expand it out but suggest they don't prove it since it seems boring and tedious) }

{\it Your homework is to complete up through \autoref{exercise:discriminant-cubic-sign}. If you finish that, and still have time try the following challenge questions.}

\begin{exercise} An amazing fact about the discriminant is that it can always be written as a polynomial in terms of the coefficients, for the quadratic this was \autoref{definition:quadratic-discriminant} and \autoref{exercise:quadratic-discriminant-in-roots}. Now let's see this for \textbf{depressed cubic} (i.e. coefficient of $ x^2$ is 0). Consider the cubic
	\begin{align*}
		P(x) &= x^3 + px + q
	\end{align*}
Assume that not all three roots are the same. 
	\begin{enumerate}
		\item By definition, the {\bf critical points} of $P(x)$ (the points at which $ P(x)$ attains it's local max/min if at all) and are roots of $P'(x) = 3x^2 + p$. Find the critical points of $ P(x)$, and call them $ x_1, x_2$.
		\item Convince yourself that $ P(x)$ has repeated roots if and only if all the roots are real and one of the roots is a critical point. (Draw a graph.)
		\item Argue that if $ P(x)$ has a repeated root if and only if $ P(x_1) \cdot P(x_2) = 0$. (Compare this with \autoref{exercise:discriminant-vanishing}.)
		\item Expand $ P(x_1) \cdot P(x_2)$.
	\end{enumerate}
	Here's a fact we won't prove. It's quite easy but tedious to show this by simply expanding out the definition of the discriminant from yesterday.
	\begin{fact}
		The discriminant of the cubic $ P(x)$ is $ -27P(x_1) \cdot P(x_2) = -4p^3 - 27q^2$.
	\end{fact}
\end{exercise}


\subsection{Counting polynomials of discriminant $0$}
\begin{exercise}[Challenge 1]
	\label{exercise:repeated-roots}
	Let $p(x) = x^n + a_{n-1}x^{n-1} + \cdots + a_0$, where now the coefficients $a_i$ are
	in $\bz/p$ (i.e., take on values between $0$ and $p-1$), and the discriminant is
	also considered as a number in $\bz/p$. Show there are $p^n$ such polynomials, and exactly
	$p^{n-1}$ of them have discriminant $0$.
	Conclude that the number of squarefree polynomials of degree $n$ over $\bz/p$ is $p^n - p^{n-1}$.
	\footnote{{\it Hint:} Try factoring the polynomials over $\bz/p$, and write each polynomial
	uniquely as $f(x)g(x)^2$, where $f(x)$ is squarefree. Then count the number
of such polynomials for the degree of $f$ fixed inductively.}
\end{exercise}

\begin{exercise}[Challenge 2]
	\label{exercise:common-roots}
	Using a similar method to that of \autoref{exercise:repeated-roots},
	count the number of pairs of degree $n$ polynomials $(p,q)$ for 
	$p(x) = x^n + a_{n-1}x^{n-1} + \cdots + a_0$ and $q(x) = x^n + b_{n-1}x^{n-1} + \cdots + b_0$
	with $a_i \in \bz/p, b_i \in \bz_p$ so that $p$ and $q$ have no common root $\bmod p$.
\end{exercise}
\begin{remark}
	\label{remark:}
	If you did the two prior exercises \autoref{exercise:repeated-roots} and \autoref{exercise:common-roots} correctly, you may notice a striking similarity
between the two answers. There is in deed a deeper connection, but the answer lies deep.
Loosely speaking, if you take a polynomial $p(x)$ with no repeated roots, you can send it
to the pair of polynomials $(p(x) + p'(x), p(x))$. Here $p'(x)$ denotes the derivative of $p(x)$.
If $p(x) = \sum_i a_ix^i$ then $p'(x) = \sum_i  i \cdot a_i x^{i-1}$.
The diligent reader will check that this is a map from the space of polynomials with no repeated roots to the
space of pairs of polynomials with no repeated roots.
In some sense (which we do not explain) this map explains why the counts from 
\autoref{exercise:repeated-roots} and \autoref{exercise:common-roots} are so similar.
\end{remark}





\newpage
\section{Solving the Cubic}

	
	
	
	\subsection{Discriminant of the Cubic}
	
\todo{aaron: i feel it might be cleaner to put an exercise here reducing to solving the depressed cubic, rather than coming back to it at the end, since the reduction is so easy? i've commented it out below}
%\begin{exercise}
%	\label{exercise:}
%	Show that if $f(x) = x^3 + ax^2 +bx +c$ is any cubic polynomial with real coefficients, one can apply
%	a change of variables of the form $y = x + c$ (for $c \in \br$)
%	so that $f(y) = y^3 + py^2 + q$.
%	Such a cubic is called a {\bf depressed cubic} (meaning the coefficient of $x^2$ is $0$).
%\end{exercise}

	
	
	\subsection{Solving the Cubic}
	

	
%	We'll see later on ``why'' the discriminant can be written in terms of the coefficients $ P(x)$.
	
	\subsection{The Solution}
%		
%		
%	There is no unique way to find the roots of the cubic. 
%	We'll choose the method that is motivated by Galois theory. 
%	It'll look extremely mysterious but this is the method that generalizes to show that there is no formula for solving a general fifth degree polynomial. 
In this section, you will discover a method to solve the cubic, motivated by Galois theory.
	\begin{exercise}
	\label{exercise:coefficients_depressed_cubic}
		Let $P(x) = x^3 + px + q$ with roots $\alpha, \beta,$ and $\gamma$. Express the coefficients of $P(x)$ (namely $ 0,p$, and $q$) in terms of $\alpha$, $\beta$, and $\gamma$.
	\end{exercise}

	\begin{exercise}
	\label{exercise:cube_roots_of_unity}
	We need some identities about the \emph{cube roots of unity} before proceeding.
	\begin{enumerate}
		\item Find the three roots of the polynomial $ x^3 - 1$ over the complex numbers.
		\item Show that if $ \omega $ is a non-real root of $x^3 -1$ then the other non-real root is $ \omega^2$. Conclude that $ \overline{\omega} = \omega^2$.
			The complex numbers $\omega, \omega^2,$ and $1$ are called the \textbf{cube roots of unity}.
    \item Compute $ \omega + \omega^2$.
		\item Plot $ \omega$, $\omega^2$ on the complex plane.
		\item Find the three solutions of $ x^3 = 2$. These are the three \emph{cube roots} of $ 2$.
			\todo{aaron: why include this last part of the exercise?}
	\end{enumerate}
\end{exercise}
  The method for solving the cubic is somewhat like induction. We reduce the problem of solving the cubic to that of solving the quadratic. For this we need to find \emph{intermediate constants} which satisfy a known \emph{quadratic} and from which $\alpha,\beta,\gamma$ can be easily recovered. To this end we define
	\begin{align}
		\label{equation:intermediate_variables_cubic}
		\begin{split}
    \mu_1 &:= \beta_1 + \beta_2 \omega + \beta_3 \omega^2 \\
    \mu_2 &:= \beta_1 + \beta_2 \omega^2 + \beta_3 \omega
	\end{split}
  \end{align} 
	Our \emph{intermediate constants} are not $ \mu_1$ and $ \mu_2$ but $ \mu_1^3$ and $ \mu_2^3$.


	\begin{exercise}
		\label{exercise:intermediate_variables_cubic_1}
		Find $\alpha, \beta$ and $\gamma$ in terms of $ \mu_1,$ and $\mu_2$. \hint{Find $ \beta_1 + \beta_2$, $ \omega \beta_1 + \omega^2 \beta_2$, $ \omega^2 \beta_1 + \omega \beta_2$. }
	\end{exercise}
	


	\begin{exercise}
		\label{exercise:intermediate_variables_cubic_2}
		$ $ 
		\begin{enumerate}
			\item Use the previous exercise to find $\alpha \cdot \beta \cdot \gamma$ in terms of $ \mu_1,$ and $\mu_2$. 
				\todo{aaron: why not also ask them to find $\alpha \cdot \beta+\alpha \cdot \gamma+\beta\cdot \gamma$? dont they need to also find that to do the next part here?}
				Use this to find $ \mu_1^3 + \mu_2^3$ in terms of $ p, q$.
\hint{First expand $\alpha, \beta$.}
				\todo{aaron: can you even do this without taking a squareroot of q? I found this extremely difficult, it took about 15 minutes}
			\item Find $ \mu_1 \cdot \mu_2$ in terms of $ p,q$.
				\todo{aaron: I can't bring myself to do this part or the next, it seems so grungy. Can we maybe give them the answer and ask them to check it?}
			\item Find the coefficients of the quadratic whose roots are $ \mu_1^3$ and $ \mu_2^3$ and solve it. (Do you recognize anyone \todo{what does this mean?} here?)
		\end{enumerate}
	\end{exercise}
	
	\begin{mdframed}
		With all this work done, here's the algorithm for finding the roots of a depressed cubic $ x^3 + px + q$:
		\begin{enumerate}
			\item Find $ \mu_1^3$ and $ \mu_2^3$ by solving the quadratic you found in the previous exercise. 
			\item This does not determine $ \mu_1$ and $ \mu_2$ uniquely. Pick $ \mu_1$ as any of the three \emph{cube roots} of $ \mu_1^3$ and use Part 2 of \autoref{exercise:intermediate_variables_cubic_2} to find $ \mu_2$.
			\item Use \autoref{exercise:intermediate_variables_cubic_1} to find $ \beta_1, \beta_2, \beta_3$ in terms of $ \mu_1, \mu_2$. 
			%\item Feel happy about yourself.
		\end{enumerate}
	\end{mdframed}
		\begin{exercise}
			Use your newfound superpower to find the roots of $x^3 - 3x + 2$ and $x^3 - 3x + 1$.
		\end{exercise}
		
	\begin{exercise}
		We had to make a \emph{choice} while finding $ \mu_1$. Why does this choice not matter? Or does it? What if we had picked a different \emph{cube root} of $ \mu_1^3$?
	\end{exercise}
	
	\begin{exercise}
		Use the above algorithm to find the roots of $x^3 - 3x + 2$ and $x^3 - 3x + 1$.
	\end{exercise}
	
	What if your cubic is not depressed? Suppose the polynomial $ P(x) = x^3 + ax^2 + bx + c$ has roots $\alpha,\beta,\gamma$. 
	\begin{exercise} $ $
		\begin{enumerate}
			\item Find the coefficient of the polynomial whose roots are $\alpha+ k, \beta + k, \gamma + k$.
			\item Find a $ k$ such that the coefficient of $ x^2$ vanishes. Notice that such a $ k$ only depends on the original coefficients $ a,b$ and $c$. \todo{doesn't it only depend on $a$?}
				\todo{I'd rather put this at the begining in the form i commented out - it seems clearer as a reduction at the beginning than an tack on at the end}
			\item Do you see it now? \todo{aaron: what does this mean?}
		\end{enumerate}
	\end{exercise}
	
	\begin{remark}[The idea for solving the quartic]
		\label{remark:}
		A similar inductive technique works for the quartic, however the method is too tedious to do by hand. Suppose we're trying to find the roots $ \gamma_1, \gamma_2, \gamma_3, \gamma_4$ of a quartic \begin{align*}
			P(x) &= x^4 + a_3x^3 + a_2x^2 + a_1x + a_0
		\end{align*}
		then the strategy is to find 3 \emph{intermediate variable} $ \lambda_1, \lambda_2, \lambda_3$ such that 1) they satisfy a cubic polynomial whose coefficients can be obtained from the original coefficients and 2) the $\gamma_i's$ can recovered from the $ \lambda_i's$ `easily'. Such variables indeed exist:
			\begin{align}
				\label{equation:intermediate_variables_quartic}
				\begin{split}
					\lambda_1 &:= \gamma_1 \gamma_2 + \gamma_3 \gamma_4 \\
					\lambda_2 &:= \gamma_1 \gamma_3 + \gamma_2 \gamma_4 \\
					\lambda_3 &:= \gamma_2 \gamma_3 + \gamma_1 \gamma_4 
				\end{split}
			\end{align}
		
		Here's how you recover the $ \gamma_i$ from the $ \lambda_i$.
		\begin{exercise}
			\begin{enumerate}
				\item Convice yourself that $ \gamma_1 \gamma_2$ and $\gamma_3 \gamma_4$ are the roots of $ x^2 - \lambda_1 x + a_0$. This gives us all the $ \gamma_i \gamma_j$.
				\item Figure out a way to recover $ \gamma_i$ if you know all the $ \gamma_i \gamma_j$.
			\end{enumerate}
		\end{exercise}
		We'll later see why the $ \lambda_i$ satisfy a cubic with coefficients which can be written in terms of the $ a_i$.
	\end{remark}
	\todo{my feeling is that this is a bit much to require for a 2 chili course - it took me about an hour to do the exercises, even having known
	how to do pretty much all of them before, i think we should make this easier somehow, maybe by telling them the answers to some of the more computational questions?}
	\subsection{Optional Exercises}
	
	Solutions of the polynomial equation $ x^n = 1$ are called \textbf{$n^{th}$ roots of unity}, where $ n$ is a positive integer. An $ n^{th}$ root of unity is called \textbf{primitive} if it not an $ m^{th}$ root of unity for any $ m < n$.
	
	\begin{exercise} $ $
		\begin{enumerate}
			\item Find \todo{aaron: what does ``find`` mean?} and plot all the $ n^{th}$ roots of unity. 
			\item How many primitive $n$th roots are there? \todo{aaron: what do you want them to express their answer in terms of? the prime factorization of the number?}
		\end{enumerate}
		If $ \zeta_1, \cdots, \zeta_k$ are all the primitive $ n^{th}$ roots of unity then the polynomial $ \Phi_n(x) := \prod _{i=1}^k (x - \zeta_i)$ is called the $ n^{th}$ \textbf{cyclotomic polynomial}. Miraculously, all the coefficients of $ \Phi_n(x)$ are integers!
		\begin{enumerate}[resume]
			\item Find the cyclotomic polynomials $ \Phi_2(x), \Phi_3(x), \Phi_4(x)$, $ \Phi_5(x)$.
			\item Find the cyclotomic polynomials $ \Phi_p(x)$ where $ p$ is prime.
			\item Factor $ x^n - 1$ as a product of cyclotomic polynomials.
		\end{enumerate}
	\end{exercise}
	

\newpage
\section{Symmetry Groups}

Today we will explore symmetry groups of objects. Surprisingly, these will help us understand
how to solve cubic and quartic equations in future days.

\subsection{Symmetries of the triangle}

\begin{definition}
	\label{definition:automorphisms}
	For $X$ a subset of $\br^n$, we define the {\bf automorphisms} of $X \subset \br^n$ to be the set of
	reflections and rotations of $\br^n$ which send $X$ to $X$.
\end{definition}

\begin{exercise}
	\label{exercise:triangle-automorphisms}
	Show that the an equilateral triangle in $\br^2$ has exactly $6$ automorphisms.
	Here, we include the {\bf identity automorphism}, denoted $\id$, which fixes every point of the triangle.
	Write down these automorphisms explicitly (in terms of rotations and reflections).
\end{exercise}
\begin{exercise}
	\label{exercise:aut-is-group}
	Show that the composition of two automorphisms is again an automorphism.
	Also, show that every automorphism has an inverse. (That is, if $f: \br^n \ra \br^n$ is a reflection or rotation fixing $X$, show there is
	another automorphism $g: \br^n \ra \br^n$ fixing $X$ so that the compositions $f \circ g$ and $g \circ f$ fix every point of $X$.)
\end{exercise}
\begin{remark}
	\label{remark:}
	\autoref{exercise:aut-is-group} is another way of saying that the automorphisms of an object form a group.
\end{remark}
\begin{exercise}
	\label{exercise:r-s}
	Let $s$ denote the automorphism of the equilateral triangle which is rotation by $120\degree$
	and let $r$ denote a reflection interchanging two vertices of the equilateral triangle.
	Show that $r^2= \id,$ (where $r^2$ means $r \circ r$) $s^3 = \id$, and $rs = s^2r$.
\end{exercise}
\begin{exercise}
	\label{exercise:}
	Show that all $6$ automorphisms from \autoref{exercise:triangle-automorphisms} can be expressed as compositions
	of the elements $r$ and $s$ defined in \autoref{exercise:r-s}. In this case we say that $r$ and $s$ {\bf generate}
	the automorphism group of the equilateral triangle.
\end{exercise}

\subsection{Dihedral Groups}

We now generalize from the case of triangles to all polygons.
\begin{exercise}
	\label{exercise:}
	Show that a regular $n$-gon in $\br^2$ has $2n$ automorphisms. This set of automorphims is called the {\bf dihedral group of size $2n$}, denote $D_{2n}$.
\end{exercise}
\begin{exercise}
	\label{exercise:}
	Let $s$ denote the rotation of a regular $n$-gon by $(360/n) \degree$ about its center and $r$ denote an automorphism of a regular $n$-gon
	which is a reflection. Show that $r^2 = \id$, $s^n = \id$, and $rs = s^{n-1} r$. Show that $r$ and $s$ generate all automorphisms of the
	regular $n$-gon.
\end{exercise}
\begin{exercise}
	\label{exercise:}
	Show that $n$ of the $2n$ automorphisms of the regular $n$-gon are rotations.
	Show that the subset $D_{2n}$ of rotations is closed under composition. In this case we say that the rotations preserving the $n$-gon
	form a {\bf subgroup} of all automorphisms. This subgroup is called {\bf the cyclic group of order $n$}, denoted $C_n$.
\end{exercise}

\subsection{Symmetric Groups}
\begin{definition}
	\label{definition:}
	We define the {\bf symmetric group on $n$ elements,} $S_n$ to be the set of bijections $\left\{ 1,2,\ldots, n \right\} \ra \left\{ 1, 2, \ldots, n \right\}$
\end{definition}
Strictly speaking, the symmetric group is a little more than just this set. Given two bijections, you can compose them to get a third bijection.
\begin{exercise}
	\label{exercise:}
	Show that $S_n$ (the symmetric group on $n$ elements) has size $n!$.
\end{exercise}
\begin{exercise}
	\label{exercise:}
	Show that $S_3$ can be identified with the automorphisms of the equilateral triangle.
	\footnote{{\it Hint:} Consider how the automorphisms act on the vertices of the triangle.}
\end{exercise}

\begin{exercise}
	\label{exercise:}
	Show that the automorphisms of the tetrahedron in $\br^3$ are identified with $S_4$. \footnote{{\it Hint:} Consider the action of the automorphisms of the $4$ vertices.}
\end{exercise}
\begin{exercise}
	\label{exercise:tetrahedron-rotations}
	Show that inside the group of all automorphisms of the tetrahedron (of size $24 = 4!$, which is the size of $S_4$), there are $12$ rotations.
	Show that these rotations form a subgroup of $S_4$. This is known as the {\bf alternating group on $4$ elements}, denoted $A_4$.
\end{exercise}

{\it Your homework is to complete up through \autoref{exercise:tetrahedron-rotations}. If you finish that, and still have time try the following challenge questions.}

\begin{exercise}[Challenge 1]
	\label{exercise:}
	Inside all automorphisms of the cube, there is a subgroup of rotations. Identify that subgroup with $S_4$.
	Do the same for the octahedron.
	Show that inside the automorphisms of the cube, there is a subgroup of order $4$, whose $3$ non-identity elements consist of $180 \degree$ rotations.
	This is known as the {\bf Klein-$4$ group,} denoted $K_4$.
\end{exercise}
\begin{exercise}[Challenge 2]
	\label{exercise:}
	Determine the number of rotations of the dodecahedron. Do the same for the number of rotations of the icosahedron.
	Identify these two groups. That is, construct a bijection between these groups respecting composition.
	Show that these are subgroups of $S_5$. 
	{\it Possible hint:} For identifying this as a subgroup of $S_5$,
	one can show there are $5$ cubes which can be inscribed in a dodecahedron, and the rotations permute these cubes.
\end{exercise}




\newpage
\section{The Galois Correspondence}
	
	
	
	
	
	
	
	\subsection{Symmetric Polynomials}	
	
	Let $ P(x) = x^n + a_{n-1} x^{n-1} + \dots + a_1 x + a_0$ be a polynomial with roots $ \zeta_1, \zeta_2, \dots, \zeta_n$.
	
	\begin{exercise}
		Express the coefficients $ a_i$ of $ P(x)$ in terms of $ \zeta_i$.
	\end{exercise}
	Note that each $ a_i$ is a (multi-variable) polynomial in the roots $ \zeta_i$. These polynomials are called the \textbf{elementary symmetric polynomials}.
	The symmetric group $ S_n$ naturally \textbf{acts} on the set of polynomials in $ n$ variables by permuting the variables. 
	That is, we can ``apply'' $\sigma$ to a polynomial $Q$ to obtain the polynomial $\sigma(Q)$ defined by
		\begin{align*}
			(\sigma(Q))(\zeta_1, \zeta_2, \dots, \zeta_n) := Q(\zeta_{\sigma(1)}, \zeta_{\sigma(2)}, \dots, \zeta_{\sigma(n)})
		\end{align*}
		\begin{definition}
			\label{definition:}
			A polynomial $Q(\zeta_1, \zeta_2, \dots, \zeta_n)$ is {\bf symmetric} if for all $\sigma \in S_n $, 
	\begin{align*}
		\sigma(Q) (\zeta_1, \zeta_2, \dots, \zeta_n)= Q(\zeta_1, \zeta_2, \dots, \zeta_n)
	\end{align*}
	In other words, the polynomial remains the same after reordering the variables.
		\end{definition}
		
	\begin{exercise}
		Convince yourself that the elementary symmetric polynomials are symmetric.
	\end{exercise}
	
	Hence the `symmetric' in elementary symmetric. The reason for the `elementary' is the following theorem.
	
	\begin{theorem}
		\label{theorem:fundamental_theorem_symmetric_polynomials}
		Any symmetric polynomial $ Q(\zeta_1, \zeta_2, \dots, \zeta_n)$ can be expressed as a polynomial in the coefficients $ a_i$.
	\end{theorem}
	\todo{state as fact?}
	For example, \begin{align*}
		\zeta_1^2 + \zeta_2^2 + \dots + \zeta_n^2 
		&= (\zeta_1 + \zeta_2 + \dots + \zeta_n)^2 - 2.\zeta_1 \zeta_2 \dots \zeta_n \\
		&= a_{n-1}^2 - 2.(-1)^n a_0
	\end{align*}
	\begin{corollary}
		Any symmetric polynomial $ Q(\zeta_1, \zeta_2, \dots, \zeta_n)$ in the roots $ \zeta_i$ of a polynomial $ P(x)$ can be expressed as a polynomial in the coefficients $ a_i$ of the polynomial of $ P(x)$.
	\end{corollary}
	\todo{ask them to prove this corollary as an exercise?}
	
	We've already encountered several examples of this:
	\begin{exercise}
		Verify that for any polynomial $ P(x)$ the discriminant $ \Delta(P)$ is symmetric, and hence by \autoref{theorem:fundamental_theorem_symmetric_polynomials} can be expressed as in terms of the coefficients of $ P(x)$!
	\end{exercise}
	
	\begin{exercise} This exercise justifies why the intermediate variables for a cubic satisfy a quadratic polynomial, whose coefficients are polynomials in the original polynomial.
		\begin{enumerate}
			\item Verify that if $ P(x)$ is a cubic then $ \mu_1^3 + \mu_2^3 $ and $ \mu_1^3  \mu_2^3$ as defined in \autoref{equation:intermediate_variables_cubic} are both symmetric. 
\todo{maybe say they are allowed to assume the theorem, and use the definition of discriminant as products of differences of the roots?}
			\item Also show that $ \mu_1 + \mu_2$ is \emph{not} symmetric. This was the reason we could not use $ \mu_1, \mu_2$ as our intermediate variables.
		\end{enumerate}
	\end{exercise}
	
	\begin{exercise}
		This exercise justifies why the intermediate variables for a quartic satisfy a cubic polynomial, whose coefficients are polynomials in the original polynomial.		\begin{enumerate}
			\item Verify that if $ P(x)$ is a quartic then $ \lambda_1 + \lambda_2 + \lambda_3$, $ \lambda_1 \lambda_2 + \lambda_1 \lambda_3 + \lambda_2 \lambda_3$, and $ \lambda_1 \lambda_2 \lambda_3$ as defined in \autoref{equation:intermediate_variables_quartic} are symmetric in $ \zeta_i$. 
			\todo{at this point, people haven't seen the $\tau_i, \lambda_i$ in $2$ days, so maybe we should remind them of the definition (instead of a reference to a handout they might have lost?}
			\item Conclude that the coefficients of $ (x-\lambda_1)(x-\lambda_2)(x-\lambda_3)$ can be expressed in terms of the coefficients of $ P(x)$.
		\end{enumerate}
	\end{exercise}
	
	Hopefully now you're convinced that there was some reason behind choosing these intermediate variables. Which brings us to the question: Do such variables exist for all degrees?
	
	
	
	
	
	
	\subsection{Fixed points}
		A set of polynomials $ V = \{ Q_i(\zeta_1, \zeta_2, \dots, \zeta_n) \}_i$ is said to be \textbf{fixed point-wise} by an element $ \sigma \in S_n$ if 
			\begin{align*}
			(\sigma(Q_i))(\zeta_1, \zeta_2, \dots, \zeta_n) = Q_i(\zeta_{\sigma(1)}, \zeta_{\sigma(2)}, \dots, \zeta_{\sigma(n)})
			\end{align*}
			\todo{isn't this the definition of the action and not what it means to be fixed? I changed this above, but since you did it twice, maybe I'm wrong?}
		for every $ Q_i \in V$. By this definition, the set of symmetric polynomials is fixed by the every element of $ S_n$. 
		
		\begin{exercise}
			For a set of polynomials $ V = \{ Q_i(\zeta_1, \zeta_2, \dots, \zeta_n) \}$, prove that the subset of elements of $ S_n$ that fix $ Q$ point-wise, is itself a group (and hence a subgroup of $ S_n$). Such a subgroup is called the \textbf{stabilizer} of $ V$.
		\end{exercise}
		
		\begin{exercise} Let $ P(x)$ be a cubic polynomial. 
			\begin{enumerate}

				\item Compute $ \omega \cdot \mu_1$ and $ \omega^2\cdot \mu_1$. Similarly for $ \mu_2$.
				\item Find the subgroup of $ S_3$ which fixes the set $\{ \mu_1^3, \mu_2^3 \}$ \todo{what are these polynomials in terms of? are they in terms of the roots? or p and q?}point-wise. What do the elements not in this subgroup do to it?\hint{Don't forget that $ \omega^3 = 1$.}
				\item Find the subgroup of $ S_3$ which fixes the set $\{ \mu_1, \mu_2 \}$ point-wise.
			\end{enumerate}
		\end{exercise}
		
		\begin{exercise}
			Let $ P(x)$ be a quartic polynomial. Find the subgroup of $ S_4$ which fixes the set $ \{ \lambda_1,  \lambda_2, \lambda_3 \}$ point-wise. Have you seen this subgroup before?
		\end{exercise}
		
\todo{aaron: I'm a bit confused by the following dialogue. Aren't there basically no possible intermediate variables. I'd remove this, or at least clarify what you mean?}
		Hopefully now you see some vague pattern:\\
		Subgroups of $ S_n$ are the stabilizers of the set of intermediate variables. Which begs the question: For higher degree polynomials 
		\begin{enumerate}
			\item 
		What are the possible intermediate variables? 
	\item Can every subgroup of $ S_n$ be a stabilizer of some set of intermediate variables?
		\end{enumerate}
		
		It turns out we can answer the second question in the language of group theory.
		
		\subsection{Optional Problems}
		Problems about the proof of fundamental theorem of symmetric polynomials.
		\todo{add}



\newpage
\section{Solving the Quartic}



\newpage
\theendnotes
\end{document}


