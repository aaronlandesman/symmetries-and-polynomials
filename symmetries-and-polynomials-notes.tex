%        File: symmetries-and-polynomials-notes.tex
%     Created: Thu May 17 10:00 AM 2018 P
% Last Change: Thu May 17 10:00 AM 2018 P
%
%%%%%%%%%%%%%%%%%%%%%
%   AMS packages    %
%%%%%%%%%%%%%%%%%%%%%
% \documentclass[12 pt, reqno]{amsart}

\documentclass[reqno, 12pt, letter]{article}
\usepackage[margin=1in]{geometry}
\usepackage{mdframed}

\usepackage{hyperref}
\usepackage{etex}
\usepackage[shortlabels]{enumitem}
\usepackage{amsmath}
\usepackage{amsxtra}
\usepackage{amscd}
\usepackage{amsthm}
\usepackage{amsfonts}
\usepackage{amssymb}
\usepackage{eucal}
\usepackage[all]{xy}
\usepackage{graphicx}
\usepackage{tikz-cd}
\usepackage{mathrsfs}
\usepackage{subfiles}
\usepackage{mathpazo}
%\usepackage{euler}
\usepackage[colorinlistoftodos, textsize=tiny]{todonotes}
\usepackage{morefloats}
\usepackage{pdfpages}
\usepackage{thm-restate}
\usepackage[utf8]{inputenc}
\usepackage{epigraph}
\usepackage{csquotes}
\usepackage{gensymb}
% \usepackage[margin=1.5in]{geometry}

\graphicspath{ {images/} }

\RequirePackage{color}
\definecolor{myred}{rgb}{0.75,0,0}
\definecolor{mygreen}{rgb}{0,0.5,0}
\definecolor{myblue}{rgb}{0,0,0.65}

\usepackage{hyperref}
\hypersetup{citecolor=blue}
\usepackage{tikz}
\usetikzlibrary{matrix,arrows,decorations.pathmorphing}

\theoremstyle{plain}
\newtheorem{theorem}{Theorem}[section]
\newtheorem{proposition}[theorem]{Proposition}
\newtheorem{lemma}[theorem]{Lemma}
\newtheorem{corollary}[theorem]{Corollary}
\theoremstyle{definition}
\newtheorem{definition}[theorem]{Definition}
\newtheorem{remark}[theorem]{Remark}
\newtheorem{example}[theorem]{Example}
\newtheorem{exercise}[theorem]{Exercise}
\newtheorem{counterexample}[theorem]{Counterexample}
\newtheorem{convention}[theorem]{Convention}
\newtheorem{question}[theorem]{Question}
\newtheorem{conjecture}[theorem]{Conjecture} 
\newtheorem{goal}[theorem]{Goal}
\newtheorem{warn}[theorem]{Warning}
\newtheorem{fact}[theorem]{Fact}
\theoremstyle{remark}
\newtheorem{notation}[theorem]{Notation}
\numberwithin{equation}{section}
  
\newcommand\nc{\newcommand}
\nc\on{\operatorname}
\nc\renc{\renewcommand}
\newcommand\se{\section}
\newcommand\ssec{\subsection}
\newcommand\sssec{\subsubsection}
\newcommand\bn{{\mathbb N}}
\newcommand\bc{{\mathbb C}}
\newcommand\br{{\mathbb R}}
\newcommand\bq{{\mathbb Q}}
\newcommand\bp{{\mathbb P}}
\newcommand\CF{{\mathscr F}}
\newcommand\bz{{\mathbb Z}}
\newcommand\ba{{\mathbb A}}
\newcommand\bg{{\mathbb G}}
\newcommand\fa{{\mathfrak a}}
\newcommand\fp{{\mathfrak p}}
\newcommand\fq{{\mathfrak q}}
\newcommand\fm{{\mathfrak m}}
\newcommand\fg{{\mathfrak g}}
\newcommand\fh{{\mathfrak h}}
\newcommand\fgl{{\mathfrak {gl}}}
\newcommand\fsu{{\mathfrak {su}}}
\newcommand\fsl{{\mathfrak {sl}}}
\newcommand\fso{{\mathfrak {so}}}
\newcommand\fo{{\mathfrak {o}}}
\newcommand\fu{{\mathfrak {u}}}
\newcommand\fsp{{\mathfrak {sp}}}
\newcommand\fgsp{{\mathfrak {gsp}}}



\newcommand\sca{\mathscr A}
\newcommand\scb{\mathscr B}
\newcommand\scc{\mathscr C}
\newcommand\scd{\mathscr D}
\newcommand\sce{\mathscr E}
\newcommand\scf{\mathscr F}
\newcommand\scg{\mathscr G}
\newcommand\sch{\mathscr H}
\newcommand\sci{\mathscr I}
\newcommand\scj{\mathscr J}
\newcommand\sck{\mathscr K}
\newcommand\scl{\mathscr L}
\newcommand\scm{\mathscr M}
\newcommand\scn{\mathscr N}
\newcommand\sco{\mathscr O}
\newcommand\scp{\mathscr P}
\newcommand\scq{\mathscr Q}
\newcommand\scs{\mathscr S}
\newcommand\sct{\mathscr T}
\newcommand\scu{\mathscr U}
\newcommand\scv{\mathscr V}
\newcommand\scw{\mathscr W}
\newcommand\scx{\mathscr X}
\newcommand\scy{\mathscr Y}
\newcommand\scz{\mathscr Z}

\newcommand \ra{\rightarrow}
\newcommand \xra{\xrightarrow}
\DeclareMathOperator\spec{\text{Spec }}
\DeclareMathOperator\proj{\text{Proj }}
\DeclareMathOperator\rspec{\textit{Spec }}
\DeclareMathOperator\rproj{\textit{Proj }}
\newcommand*{\shom}{\mathscr{H}\kern -.5pt om}
\newcommand*{\stor}{\mathscr{T}\kern -.5pt or}
\newcommand*{\sext}{\mathscr{E}\kern -.5pt xt}
\newcommand \mg{{\mathscr M_g}}

\makeatletter
\newcommand{\customlabel}[2]{\protected@write \@auxout {}{\string \newlabel {#1}{{#2}{\thepage}{#2}{#1}{}} }\hypertarget{#1}{#2}}
\newcommand\sh[1]{\sco_{#1}^{\rm{sh}}}
\DeclareMathOperator\id{id}
\DeclareMathOperator\tor{Tor}
\renewcommand\hom{Hom}
\DeclareMathOperator\coker{coker}
\DeclareMathOperator\ord{ord}
\DeclareMathOperator\hilb{Hilb}
\DeclareMathOperator\rk{rk}
\DeclareMathOperator\di{div}
\DeclareMathOperator\pic{Pic}
\DeclareMathOperator\lcm{lcm}
\DeclareMathOperator\rank{rank}
\DeclareMathOperator\codim{codim}
\DeclareMathOperator\vol{Vol}
\DeclareMathOperator\supp{Supp}
\DeclareMathOperator\spn{Span}
\DeclareMathOperator\im{im}
\DeclareMathOperator\End{End}
\DeclareMathOperator\sym{Sym}
\DeclareMathOperator\pgl{PGL}
\DeclareMathOperator\sat{Sat}
\DeclareMathOperator\blow{Bl}
\renewcommand\sp{\mathrm{Sp}}
\DeclareMathOperator\gsp{GSp}
\DeclareMathOperator\sgn{sgn}
\DeclareMathOperator\gal{gal}
\DeclareMathOperator\tr{tr}
\renewcommand\char{char}
\newcommand\bbf{{\mathbb F}}
\newcommand\bk{{\Bbbk}}
\newcommand\ul{\underline}
\newcommand\ol{\overline}
\DeclareMathOperator\pr{pr}
\DeclareMathOperator\ev{ev}
\DeclareMathOperator\maj{maj}
\DeclareMathOperator\inv{inv}
\DeclareMathOperator\isom{isom}
\DeclareMathOperator\mor{mor}
\DeclareMathOperator\aut{Aut}
\DeclareMathOperator\gl{GL}
\renewcommand\sl{\mathrm{SL}}
\DeclareMathOperator\mat{Mat}
\DeclareMathOperator\stab{Stab}
\DeclareMathOperator\so{SO}
\DeclareMathOperator\su{SU}
\renewcommand\u{\mathrm{U}}
\DeclareMathOperator\lie{Lie}
\DeclareMathOperator\ad{ad}
\DeclareMathOperator\Ad{Ad}
\renewcommand\o{{\rm{O}}}

\renewcommand{\thefootnote}{\fnsymbol{footnote}}
%\newcommand{\hint}[1]{\footnote{\raggedleft\rotatebox{180}{\tiny{{Hint:} #1\hfill}}}}
\newcommand{\hint}[1]{\footnote{{Hint:} #1\hfill}}


\setcounter{MaxMatrixCols}{20}

\def\listtodoname{List of Todos}
\def\listoftodos{\@starttoc{tdo}\listtodoname}

\usepackage{endnotes}

\let\footnote=\endnote



\title{Symmetries and Polynomials Notes}
\author{Aaron Landesman and Apurva Nakade}

\usepackage{microtype}
\begin{document}

\maketitle

\section{The Discriminant}

\subsection{Quadratic Polynomials}
\begin{definition}[Quadratic discriminant]
	\label{definition:quadratic-discriminant}
	Let $P(x) = x^2 + bx + c$ be a polynomial with $b$ and $c$ real numbers.
	The discriminant $\Delta(P)$ is by definition $b^2 - 4c$.
\end{definition}
\begin{exercise}
	\label{exercise:}
	If the polynomial $P(x) = x^2 + bx + c$ has roots $\alpha$ and $\beta$, express $b$ and $c$
	in terms of $\alpha$ and $\beta$.
\end{exercise}
\begin{exercise}
	\label{exercise:quadratic-roots}
	What does the sign of the discriminant (i.e., whether $\Delta(P) > 0, < 0$ or $=0$) 
	tell you about the roots $\alpha$ and $\beta$?
	\footnote{{\it Hint:} If the discriminant is $>0$ show that the roots are real, if it is equal
	to $0$, show they are the same, if it is less than $0$, show they are complex numbers
which are not real.}
\end{exercise}
\begin{exercise}
	\label{exercise:quadratic-discriminant-in-roots}
	Express the discriminant of the polynomial $P(x) = x^2 + bx + c$ in terms of
	the roots $\alpha, \beta$.
\end{exercise}

\subsection{The discriminant in general}
\begin{definition}
	\label{definition:discriminant}
	For $P(x)$ a polynomial of the form $P(x) = (x-r_1)(x-r_2) \cdots (x-r_n)$,
	define the discriminant $\Delta(P) := \prod_{1 \leq i < j \leq n} (r_i - r_j)^2$.
\end{definition}
\begin{exercise}
	\label{exercise:}
	Verify that for $P(x)$ of degree $2$, the definition of the discriminant
	of a general polynomial from \autoref{definition:discriminant} agrees with
	that of a quadratic polynomial given in \autoref{definition:quadratic-discriminant}.
	\footnote{{\it Hint:} Use \autoref{exercise:quadratic-discriminant-in-roots}.}
\end{exercise}
The following exercise shows that the discriminant measures, in a precise sense, how far apart
the roots of a polynomial are.
\begin{exercise}
	\label{exercise:discriminant-vanishing}
	Show that for $P$ a polynomial, $\Delta(P) = 0$ if and only if $P$ has a repeated root.
\end{exercise}
\subsection{Cubic discriminants}
\begin{exercise}
	\label{exercise:real-root-cubic}
	Show that a cubic polynomial with real coefficients $P(x) = x^3 + ax + bx + c$
	always has a real root.
	\footnote{{\it Hint:} Graph the cubic and show it intersects the line $P(x) = 0$.}
\end{exercise}
\begin{exercise}
	\label{exercise:}
	Using \autoref{exercise:real-root-cubic}, show that a cubic polynomial either has
	$3$ real roots or $2$ complex conjugates roots (of the form $a + bi, a-bi$ for $a,b$ real numbers) and one real root. 
	\footnote{{\it Hint:} Factor out the real root, and use your understanding of quadratic polynomials.}
\end{exercise}
\begin{exercise}
	\label{exercise:discriminant-cubic-sign}
	For $P(x) = x^3 + ax^2 + bx + c$ a cubic polynomial with real coefficients, show $\Delta(P) = 0$ if and only 
	if there is a repeated root, $\Delta(P) > 0$ if and only if $P$ has three distinct real roots, and $\Delta(P) < 0$ if and only if $P$ has two complex conjugate roots and one real root.
	Compare your answer to \autoref{exercise:quadratic-roots}.
\end{exercise}

{\it Your homework is to complete up through \autoref{exercise:discriminant-cubic-sign}. If you finish that, and still have time try the following challenge questions.}

\begin{exercise}[Challenge 1]
	\label{exercise:cubic-discriminant}
	An amazing fact about the discriminant is that it can always be written as a polynomial in terms of the coefficients, for the quadratic this was \autoref{definition:quadratic-discriminant} and \autoref{exercise:quadratic-discriminant-in-roots}. Now let's see this for a \textbf{depressed cubic} (i.e. coefficient of $ x^2$ is 0). Consider the cubic
	\begin{align*}
		P(x) &= x^3 + px + q
	\end{align*}
Assume that not all three roots are the same. 
	\begin{enumerate}
		\item By definition, the {\bf critical points} of $P(x)$ (the points at which $ P(x)$ attains it's local max/min if at all) and are roots of $P'(x) = 3x^2 + p$. Find the critical points of $ P(x)$, and call them $ x_1, x_2$.
		\item Convince yourself that $ P(x)$ has repeated roots if and only if all the roots are real and one of the roots is a critical point. (Draw a graph.)
		\item Argue that $ P(x)$ has a repeated root if and only if $ P(x_1) \cdot P(x_2) = 0$. (Compare this with \autoref{exercise:discriminant-vanishing}.)
		\item Expand $ P(x_1) \cdot P(x_2)$.
	\end{enumerate}
	Here's a fact we won't prove. It's quite easy but tedious to show this by simply expanding out the definition of the discriminant from yesterday.
	\begin{fact}
		The discriminant of the cubic $ P(x)$ is $ -27P(x_1) \cdot P(x_2) = -4p^3 - 27q^2$.
	\end{fact}
\end{exercise}

\subsection{Counting polynomials of discriminant $0$}
The following questions are quite tricky, but fun. Only attempt them if you've already solved \autoref{exercise:cubic-discriminant}.
\begin{exercise}[Challenge 2]
	\label{exercise:repeated-roots}
	Let $P(x) = x^n + a_{n-1}x^{n-1} + \cdots + a_0$, where now the coefficients $a_i$ are
	in $\bz/p$ (i.e., take on values between $0$ and $p-1$), and the discriminant is
	also considered as a number in $\bz/p$. To make sense of discriminant, you may assume that
	every such polynomial factors uniquely as a product of irreducible polynomials with coefficients in $\bz/p$.
	Show there are $p^n$ such polynomials, and exactly
	$p^{n-1}$ of them have discriminant $0$.
	Conclude that the number of squarefree polynomials of degree $n$ over $\bz/p$ is $p^n - p^{n-1}$.
	\footnote{{\it Hint:} Try factoring the polynomials over $\bz/p$, and write each polynomial
	uniquely as $f(x)g(x)^2$, where $f(x)$ is squarefree. Then count the number
of such polynomials for the degree of $f$ fixed inductively.}
\end{exercise}

\begin{exercise}[Challenge 3]
	\label{exercise:common-roots}
	Using a similar method to that of \autoref{exercise:repeated-roots},
	count the number of pairs of degree $n$ polynomials $(P,Q)$ for 
	$P(x) = x^n + a_{n-1}x^{n-1} + \cdots + a_0$ and $Q(x) = x^n + b_{n-1}x^{n-1} + \cdots + b_0$
	with $a_i \in \bz/p, b_i \in \bz_p$ so that $P$ and $Q$ have no common root $\bmod p$.
\end{exercise}
\begin{remark}
	\label{remark:}
	If you did the two prior exercises \autoref{exercise:repeated-roots} and \autoref{exercise:common-roots} correctly, you may notice a striking similarity
between the two answers. There is in deed a deeper connection, but the answer lies deep.
Loosely speaking, if you take a polynomial $P(x)$ with no repeated roots, you can send it
to the pair of polynomials $(P(x) + P'(x), P(x))$. Here $P'(x)$ denotes the derivative of $P(x)$.
If $P(x) = \sum_i a_ix^i$ then $P'(x) = \sum_i  i \cdot a_i x^{i-1}$.
The diligent reader will check that this is a map from the space of polynomials with no repeated roots to the
space of pairs of polynomials with no repeated roots.
In some sense (which we do not explain) this map explains why the counts from 
\autoref{exercise:repeated-roots} and \autoref{exercise:common-roots} are so similar.
\end{remark}



\newpage
\section{Solving the Cubic}
	In this section, you will discover a method to solve the cubic, motivated by Galois theory.
	
	\begin{exercise}
		\label{exercise:}
		Show that if $f(x) = x^3 + ax^2 +bx +c$ is any cubic polynomial with real coefficients, one can apply
		a change of variables of the form $x = y + c$ (for $c \in \br$)
		so that $f(y) = y^3 + py^2 + q$.
		As we've already seen before such a cubic is called a {\bf depressed cubic} (meaning the coefficient of $x^2$ is $0$).
	\end{exercise}

	From now on we'll work with $P(x) = x^3 + px + q$ with roots $ r_1, r_2, r_3$.
	
	\begin{exercise}
	\label{exercise:coefficients_depressed_cubic}
	Express the coefficients of $P(x)$ (namely $ 0,p$, and $q$) in terms of $ r_1, r_2, r_3$.
	\end{exercise}

	\begin{exercise}
	\label{exercise:cube_roots_of_unity}
	We need some identities about the \emph{cube roots of unity} before proceeding.
	\begin{enumerate}
		\item Find the three roots of the polynomial $ x^3 - 1$ over the complex numbers.
		\item Show that if $ \omega $ is a non-real root of $x^3 -1$ then the other non-real root is $ \omega^2$. Conclude that $ \overline{\omega} = \omega^2$.
			The complex numbers $\omega, \omega^2,$ and $1$ are called the \textbf{cube roots of unity}.
    \item Compute $ \omega + \omega^2$.
		\item Plot $ \omega$, $\omega^2$ on the complex plane.
	\end{enumerate}
\end{exercise}
  The method for solving the cubic is somewhat like induction. We reduce the problem of solving the cubic to that of solving the quadratic. For this we need to find \emph{intermediate constants} which satisfy a known \emph{quadratic} and from which $r_1,r_2,r_3$ can be easily recovered. To this end we define
	\begin{align}
		\label{equation:intermediate_variables_cubic}
		\begin{split}
    \mu_1 &:= r_1 + r_2 \omega + r_3 \omega^2 \\
    \mu_2 &:= r_1 + r_2 \omega^2 + r_3 \omega
	\end{split}
  \end{align} 
	Our \emph{intermediate constants} are not $ \mu_1$ and $ \mu_2$ but $ \mu_1^3$ and $ \mu_2^3$.


	\begin{exercise}
		\label{exercise:intermediate_variables_cubic_1}
		Verify that
		\begin{align*}
			r_1 = \dfrac{\mu_1 + \mu_2}{3} 
			\mbox { , } r_2 = \dfrac{\omega^2 \mu_1 + \omega \mu_2}{3}
			\mbox { , } r_3 = \dfrac{\omega \mu_1 + \omega^2 \mu_2}{3}
		\end{align*}
		are the solutions to \autoref{equation:intermediate_variables_cubic} and $ r_1 + r_2 + r_3 = 0$. (These are in fact unique solutions.)
	\end{exercise}
	


	\begin{exercise}
		\label{exercise:intermediate_variables_cubic_2}
		$ $ 
		\begin{enumerate}
			\item Show that $27r_1  r_2  r_3 = \mu_1^3 + \mu_2^3$. What is this in terms of $ p, q$?
			\item Show that $ \mu_1 \mu_2 = -3p$.
			\item Conclude that $ \mu_1^3, \mu_2^3$ are the roots of the quadratic $ x^2 + 27qx - 27p$. Solve it to find $ \mu_1^3, \mu_2^3$. (Do you recognize anything in the solution?)
		\end{enumerate}
	\end{exercise}
	
	\begin{mdframed}
		With all this work done, here's the algorithm for finding the roots of a depressed cubic $ x^3 + px + q$:
		\begin{enumerate}
			\item Find $ \mu_1^3$ and $ \mu_2^3$ as in \autoref{exercise:intermediate_variables_cubic_2}.
			\item This does not determine $ \mu_1$ and $ \mu_2$ uniquely. Pick $ \mu_1$ as any of the three \emph{cube roots} of $ \mu_1^3$ and use Part 2 of \autoref{exercise:intermediate_variables_cubic_2} to find $ \mu_2$.
			\item Use \autoref{exercise:intermediate_variables_cubic_1} to find $ r_1, r_2, r_3$ in terms of $ \mu_1, \mu_2$. 
			%\item Feel happy about yourself.
		\end{enumerate}
	\end{mdframed}
		\begin{exercise}
			Use this method to find the roots of $x^3 - 3x + 2$ and $x^3 - 3x + 1$.
		\end{exercise}
		
	\begin{exercise}
		We had to make a \emph{choice} while finding $ \mu_1$. Why does this choice not matter? Or does it? What if we had picked a different \emph{cube root} of $ \mu_1^3$?
	\end{exercise}
	
	\begin{remark}[The idea for solving the quartic]
		\label{remark:solving_the_quartic}
		A similar inductive technique works for the quartic, however the method is too tedious to do by hand. Suppose we're trying to find the roots $ r_1, r_2, r_3, r_4$ of a quartic \begin{align*}
			P(x) &= x^4 + a_3x^3 + a_2x^2 + a_1x + a_0
		\end{align*}
		then the strategy is to find 3 \emph{intermediate variable} $ \lambda_1, \lambda_2, \lambda_3$ such that 1) they satisfy a cubic polynomial whose coefficients can be obtained from the original coefficients and 2) the $r_i's$ can recovered from the $ \lambda_i's$ ``easily''. Such variables indeed exist:
			\begin{align}
				\label{equation:intermediate_variables_quartic}
				\begin{split}
					\lambda_1 &:= r_1 r_2 + r_3 r_4 \\
					\lambda_2 &:= r_1 r_3 + r_2 r_4 \\
					\lambda_3 &:= r_2 r_3 + r_1 r_4 
				\end{split}
			\end{align}
			
		{\it Your homework is to complete up through \autoref{remark:solving_the_quartic}. If you finish that, and still have time try the following challenge questions.}
			
		\begin{exercise}[Challenge 1]
			Here's how you recover the $ r_i$ from the $ \lambda_i$.
			\begin{enumerate}
				\item Convice yourself that $ r_1 r_2$ and $r_3 r_4$ are the roots of $ x^2 - \lambda_1 x + a_0$. This gives us all the $ r_i r_j$.
				\item Figure out a way to recover $ r_i$ if you know all the $ r_i r_j$.
			\end{enumerate}
		\end{exercise}
		We'll later see why the $ \lambda_i$ satisfy a cubic with coefficients which can be written in terms of the $ a_i$.
	\end{remark}
	
	
	
	
	
	
	\subsection{Roots of Unity}
		Solutions of the polynomial equation $ x^n = 1$ are called \textbf{$n^{th}$ roots of unity}, where $ n$ is a positive integer. An $ n^{th}$ root of unity is called \textbf{primitive} if it not an $ m^{th}$ root of unity for any $ m < n$.
	
	\begin{exercise}[Challenge 2] $ $
		\begin{enumerate}
			\item Show that the $ n^{th}$ roots of unity are $ e^{2 \pi i k / n}$ where $ 0 \le k < n$ is a positive integer. Plot them on the complex plane. 
			\item Show that there are exactly $ \phi(n)$ primitive $ n^{th}$ roots of unity, where $ \phi(n)$ is the number of positive integers less than $ n$ relatively prime to it.
		\end{enumerate}
		If $ \zeta_1, \dots, \zeta_k$ are all the primitive $ n^{th}$ roots of unity then the polynomial $ \Phi_n(x) := \prod _{i=1}^k (x - \zeta_i)$ is called the $ n^{th}$ \textbf{cyclotomic polynomial}. Miraculously, all the coefficients of $ \Phi_n(x)$ are integers!
		\begin{enumerate}[resume]
			\item Find the cyclotomic polynomials $ \Phi_2(x), \Phi_3(x), \Phi_4(x)$, $ \Phi_5(x)$.
			\item Find the cyclotomic polynomials $ \Phi_p(x)$ where $ p$ is prime.
			\item Factor $ x^n - 1$ as a product of cyclotomic polynomials.
		\end{enumerate}
	\end{exercise}
	

\newpage
\section{Symmetry Groups}

Today we will explore symmetry groups of objects. Surprisingly, these will help us understand
how to solve cubic and quartic equations in future days.

\subsection{Symmetries of the triangle}

\begin{definition}
	\label{definition:automorphisms}
	For $X$ a subset of $\br^n$, we define the {\bf automorphisms} of $X \subset \br^n$ to be the set of
	reflections and rotations of $\br^n$ which send $X$ to $X$.
\end{definition}

\begin{exercise}
	\label{exercise:triangle-automorphisms}
	Show that an equilateral triangle in $\br^2$ has exactly $6$ automorphisms.
	Here, we include the {\bf identity automorphism}, denoted $\id$, which fixes every point of the triangle.
	Write down these automorphisms explicitly (in terms of rotations and reflections).
\end{exercise}
\begin{remark}
	\label{remark:aut-is-group}
	Note that the composition of two automorphisms is again an automorphism.
	Also, automorphism has an inverse, because you can simply ``undo'' the rotation or reflection.
	This makes the set of automorphisms into what is called a {\bf group}. That is,
	a group is a set with a composition operation (that is associative), which has an identity and inverses.
\end{remark}
\begin{exercise}
	\label{exercise:r-s}
	Let $s$ denote the automorphism of the equilateral triangle which is rotation by $120\degree$
	and let $r$ denote a reflection interchanging two vertices of the equilateral triangle.
	Show that $r^2= \id,$ (where $r^2$ means $r \circ r$) $s^3 = \id$, and $rs = s^2r$ (here $rs$ means you first apply $r$, then apply $s$).
\end{exercise}
\begin{exercise}
	\label{exercise:}
	Show that all $6$ automorphisms from \autoref{exercise:triangle-automorphisms} can be expressed as compositions
	of the elements $r$ and $s$ defined in \autoref{exercise:r-s}. In this case we say that $r$ and $s$ {\bf generate}
	the automorphism group of the equilateral triangle.
\end{exercise}

\subsection{Dihedral Groups}

We now generalize from the case of triangles to all polygons.
\begin{exercise}
	\label{exercise:}
	A regular $n$-gon in $\br^2$ has $2n$ automorphisms, what are they? 
	%\hint{There are $n$ rotations by $(k * 360/n)\degree$ for $0 \leq k \leq n-1$ and $n$ reflections.} 
	This set of automorphims is called the {\bf dihedral group of size $2n$}, denote $D_{2n}$.
\end{exercise}
\begin{exercise}
	\label{exercise:}
	Let $s$ denote the rotation of a regular $n$-gon by $(360/n) \degree$ about its center and $r$ denote an automorphism of a regular $n$-gon
	which is a reflection. Show that $r^2 = \id$, $s^n = \id$, and $rs = s^{n-1} r$. Show that $r$ and $s$ generate all automorphisms of the
	regular $n$-gon by showing the $2n$ elements are $1, s, s^2, \ldots, s^{n-1}, r, rs, \ldots, rs^{n-1}$.
	\todo{Apurva: May be give them the $ 2n$ elements in terms of $ r,s$ and ask them to check? The reflections are a little hard to express in terms of $ rs$.}
	\todo{aaron: Sure. I did it. The reflections are all $rs^k$. Is that hard to express?}
\end{exercise}
\begin{remark}
	\label{remark:}
	Of the $2n$ automorphisms of the regular $n$-gon, $ n$ are rotations. 
	It is easy to see that the subset of rotations is closed under composition. 
	In this case we say that the rotations preserving the regular $n$-gon
	form a {\bf subgroup} of all automorphisms. 
	This subgroup is called {\bf the cyclic group of order $n$}, denoted $C_n$.
\end{remark}

\subsection{Symmetric Groups}
\begin{definition}
	\label{definition:}
	We define the {\bf symmetric group on $n$ elements,} $S_n$ to be the set of bijections $\left\{ 1,2,\ldots, n \right\} \ra \left\{ 1, 2, \ldots, n \right\}$
\end{definition}
Strictly speaking, the symmetric group is a little more than just this set. Given two bijections, you can compose them to get a third bijection.
This makes the set of these bijections into a group.
\begin{exercise}
	\label{exercise:}
	Show that $S_n$ (the symmetric group on $n$ elements) has size $n!$.
\end{exercise}
\begin{exercise}
	\label{exercise:}
	Show that $S_3$ can be identified with the automorphisms of the equilateral triangle.
	\footnote{{\it Hint:} Consider how the automorphisms act on the vertices of the triangle.}
\end{exercise}

\begin{exercise}
	\label{exercise:}
	Show that the automorphisms of the tetrahedron in $\br^3$ are identified with $S_4$. \footnote{{\it Hint:} Consider the action of the automorphisms of the $4$ vertices.}
\end{exercise}
\begin{exercise}
	\label{exercise:tetrahedron-rotations}
	Show that inside the group of all automorphisms of the tetrahedron (of size $24 = 4!$, which is the size of $S_4$), there are $12$ rotations.
	Show that these rotations form a subgroup of $S_4$. This is known as the {\bf alternating group on $4$ elements}, denoted $A_4$.
\end{exercise}

{\it Your homework is to complete up through \autoref{exercise:tetrahedron-rotations}. If you finish that, and still have time try the following challenge questions.}

\begin{exercise}[Challenge 1]
	\label{exercise:cube}
	Inside all automorphisms of the cube, there is a subgroup of rotations. 
	What is the size of this group? Can you identify this group? 
	\hint{Look at the long diagonals. }
\end{exercise}

\begin{exercise}[Challenge 2]
	\label{exercise:octahedron}
	Similarly, identify the group of automorphisms of the octahedron with $S_4$.
%	Show that inside the automorphisms of the cube, there is a subgroup of order $4$, whose $3$ non-identity elements consist of $180 \degree$ rotations.
%	This is known as the {\bf Klein-$4$ group,} denoted $K_4$.
	\hint{Look at the four diagonals joining opposite sides. Alternatively, use \autoref{exercise:cube} and that the octahedron ``dual'' to the cube (the faces of the cube correspond to the vertices of the octahedron and the faces of the octahedron correspond to the vertices of the cube).}
\end{exercise}

\begin{exercise}[Challenge 3]
	\label{exercise:dodecahedron}
	Determine the number of rotations of the dodecahedron. Do the same for the number of rotations of the icosahedron.
	Identify these two groups. That is, construct a bijection between these groups respecting composition.
	Show that these are subgroups of $S_5$. 
	{\it Possible hint:} For identifying this as a subgroup of $S_5$,
	one can show there are $5$ cubes which can be inscribed in a dodecahedron, and the rotations permute these cubes.
\end{exercise}

\newpage
\section{Commutators and symmetric polynomials}
Today we discuss commutators of groups in order to apply it to solving the cubic and quartic.
Tomorrow we will explain how these commutators let us solve the cubic and quartic equations.
\todo{Apurva: I think this section is hard as we're defining a new concept. (This is the only section where we're doing this.) We can have far fewer problems here (5-6) to let them digest the definition. We can get rid of 4.2 and replace it one problem on $ K_4$ and a simplified version of Challenge Problem 1.}
\begin{definition}
	\label{definition:}
	Let $G$ be a group. The {\bf commutator subgroup} of $G$, denoted $\left[ G,G \right] \subset G$ is the subgroup of $G$ generated
	by elements of the form $ghg^{-1}h^{-1}$ for $g, h$ elements of $G$.
	Here $g^{-1}$ denotes the inverse of $g$.
\end{definition}
\begin{remark}
	\label{remark:}
	We will mostly be concerned with the case that $G$ is the symmetric group, or some group of automorphisms of a subset $X \subset \br^n$.
	In the case of $X \subset \br^n$, if $g$ and $h$ are two automorphisms of $G$, $ghg^{-1}h^{-1}$ means that you first apply $g$, then apply $h$, then
	``undo'' $g$ and then ``undo'' $h$.
\end{remark}
\begin{exercise}
	\label{exercise:}
	Let $G = S_3$ (it may help to think about this as automorphisms of an equilateral triangle).
	Let $r$ denote the reflection (in terms of $S_3$ this sends $1 \mapsto 2, 2 \mapsto 1, 3\mapsto 3$)
	and $s$ denote rotation by $120 \degree$ (in $S_3$ this sends $1 \mapsto 2, 2 \mapsto 3, 3 \mapsto 1$).
	Show that $rsr^{-1}s^{-1} = s^2$.
	Show that the commutator $\left[ S_3, S_3 \right]$ is generated by $s^2 = rsr^{-1}s^{-1}$, and hence is exactly the subgroup
	with elements $\left\{ \id, s, s^2 \right\}$.
\end{exercise}
\begin{exercise}
	\label{exercise:}
	Let $G = C_3$ (the group of rotations of the triangle). Show that the commutator just the identity automorphism, $\left[ C_n, C_n \right] = \left\{ \id \right\}$.
	In other words, show that for any $g, h \in C_3$, $ghg^{-1}h^{-1} = \id$.
	More generally, show that if $G = C_n$, $\left[ C_n, C_n \right] = \left\{ \id \right\}$.
	\todo{Apurva: we could also add an exercise to show the same for $ K_4$.}
\end{exercise}
%\begin{exercise}
%	\label{exercise:}
%	Let $G = D_{2n},$ the automorphisms of the regular $n$-gon.
%	For $r$ a reflection and $s$ a rotation by $(360/n) \degree$, show
%	Show that $rsr^{-1}s^{-1} = s^2$.
%	Show that the commutator $\left[ D_{2n}, D_{2n} \right]$ is generated by $s^2 = rsr^{-1}s^{-1}$
%	(i.e., it is all powers of $s^2$).
%	Depending on whether $n$ is even or odd, determine the size of $\left[ D_{2n}, D_{2n} \right]$.
%\end{exercise}

\subsection{Alternating Groups}

\begin{definition}
	\label{definition:}
	The $n$th {\bf alternating group}, denoted $A_n$ is by definition $A_n := \left[ S_n, S_n \right]$, the commutator of $S_n$.
\end{definition}
\begin{exercise}[Reality check]
	\label{exercise:low-alternating-groups}
	Verify that $A_3 = C_3$.	
\end{exercise}

\subsection{Commutators of $S_4$}
\todo{Apurva: this entire section is too hard. It's very time consuming to compute compositions of permutations. Which means computing commutators is even harder. This section would go beyond 3 chili. We have to give $A_4 = [S_4,S_4]$ and $ K_4=[A_4,A_4]$ as a fact.}

We next compute commutators of $S_4$.
To this end, it will be useful to have cycle notation.

\begin{exercise}
	\label{exercise:}
	Define a ``cycle notation'' for elements in $S_n$. See \autoref{figure:cycle-notation} on the next page for a pictorial discription.
			\begin{figure}[h!]
				% \includegraphics[scale=1.5]{cycle-1.png}	
				% \includegraphics[scale=1.5]{cycle-2.png}	
				\caption{This is a depiction of two elements thought of as elements of $S_8$. The first corresponds the the permutation fixing $2$ and $5$,
				and sending $1 \mapsto 4, 4 \mapsto 6, 6 \mapsto 8, 8\mapsto 3, 3 \mapsto 7, 7\mapsto 1$.
			The corresponding cycle notation for the first picture is $(146837)(2)(5)$ (where each parenthesized group of numbers
			correspond to a cycle in the above diagram. Similarly, the second permutation has cycle notation $(14625837)$.
			}
\label{figure:cycle-notation}\end{figure}
\end{exercise}
\begin{remark}
	\label{remark:}
	If $s \in S_n$ is an element with certain cycle notation, we often omit all singletons from the cycle notation.
	So, for example, we denote the element of $S_8$ with cycle notation $(146837)(2)(5)$ simply as $(146837)$.
\end{remark}

You may prove the following facts, or assume it if you'd like:
\begin{fact}
	\label{fact:transpositions}
	\begin{enumerate}
		\item The symmetric group is generated by elements with cycle notation of the form $(ij)$. These elements are called {\bf transpositions}. That is, transpositions switch (transpose) two numbers in $\left\{ 1, \ldots, n \right\}$ and preserve the rest.
		\item If a group is generated by a collections of elements, the commutator subgroup is generated by commutators of those elements.
	\end{enumerate}
\end{fact}
\begin{exercise}
	\label{exercise:commutators-generate}
	Using \autoref{fact:transpositions}, show that $A_n$ is generated by commutators of transpositions.
\end{exercise}

\begin{exercise}
	\label{exercise:}
	Compute $A_4$ as a subgroup of $S_4$ in the following steps:
	\begin{enumerate}
		\item Show that $(123)(4)$ is the commutator of $(12)$ (the transposition switching $1$ and $2$) and $(13)$. 
		\item Show that in general, the commutator of two transpositions is either a $3$-cycle when the transpositions have one element in common
			(i.e., an element with cycle notation of the form $(abc)$
			for $a,b,c \in \left\{ 1,2,3,4 \right\}$ distinct integers) or it is the identity permutation.
		\item Conclude from \autoref{exercise:commutators-generate} that $3$-cycles generate $A_4$.
		\item Compute the set of elements in $A_4$. Show it has size $12$.
			\hint{You should get that $A_4$ contains all three cycles, together with the four elements $\id, (12)(34), (13)(24), (14)(23)$.}
	\end{enumerate}
\end{exercise}
\begin{exercise}
	\label{exercise:klein-4}
	Compute the commutator $\left[ A_4, A_4 \right]$. Show it has order $4$ and is explicitly given by the element $\id, (12)(34), (13)(24), (14)(23)$.
	This is denoted $K_4$ and is called the {\bf Klein-$4$ group}.
\end{exercise}

{\it Your homework is to complete up to \autoref{exercise:klein-4}. If you have time, attempt the following challenge
problems}

\subsection{Geometric computations of commutators}
We now give some geometric ways to see various commutator subgroups of $S_4$.
\begin{exercise}[Challenge 1]
		\todo{Apurva: I do not know what the proof of this is, but this is a great problem. If you can outline a handwave-y proof/idea of this, it would fit the theme of previous section really well. You can ask them to verify instead of prove? Dunno if that's any better.}
	\label{exercise:}
	Using that the automorphisms of the tetrahedron can be identified with $S_4$ from \autoref{exercise:tetrahedron-rotations}, 
	show that the rotations of the tetrahedron can be identified with $A_4$.
	Describe the Klein-$4$ group in terms of rotations of the tetrahedron. Can you use this description to show $K_4 = \left[ A_4, A_4 \right]$?
\end{exercise}
\begin{exercise}[Challenge 2]
	\label{exercise:}
	Recall that in \autoref{exercise:cube}, we identified the rotations of the cube with $S_4$.
	Geometrically describe which rotations lie in the subgroup $A_4 \subset S_4$.
	Geometrically describe which rotations lie in the subgroup $K_4 \subset S_4$.
	Can you use these geometric descriptions to verify $A_4 = \left[ S_4, S_4 \right]$ and $K_4 = \left[ A_4, A_4 \right]$?
\end{exercise}
\begin{exercise}[Challenge 3]
	\label{exercise:}
	Given four distinct ordered complex numbers $a,b,c,d$, the {\bf cross ratio} is defined as
	\begin{align*}
		\mathrm{r}(a,b;c,d) := \frac{(c-a)(d-b)}{(c-b)(d-a)}.
	\end{align*}
	Note that $S_4$ acts on the set $\left\{ a,b,c,d \right\}$ by permuting them. Show that $K_4$ preserves the cross ratio.
	That is, if $\sigma \in K_4$ then then $\mathrm{r}(a,b;c,d) = \mathrm{r}(\sigma(a),\sigma(b);\sigma(c),\sigma(d))$.
	Find an example of some set of four complex numbers $(a,b,c,d)$ for which $K_4$ is exactly the subgroup that preserves
	the cross ratio.
\end{exercise}



%\begin{exercise}[Challenge]
%	\label{exercise:}
%	Show in general that $|A_n| = |S_n|/2 = n!/2$ in the following steps:
%	\begin{enumerate}
%		\item Define a ``cycle notation'' for elements in $S_n$. See \autoref{figure:cycle-notation} for a pictorial discription.
%			\begin{figure}[H]
%				\includegraphics[scale=1.5]{cycle-1.png}	
%				\includegraphics[scale=1.5]{cycle-2.png}	
%				\caption{This is a depiction of two elements thought of as elements of $S_8$. The first corresponds the the permutation fixing $2$ and $5$,
%				and sending $1 \mapsto 4, 4 \mapsto 6, 6 \mapsto 8, 8\mapsto 3, 3 \mapsto 7, 7\mapsto 1$.
%			The corresponding cycle notation for the first picture is $(146837)(2)(5)$ (where each parenthesized group of numbers
%			correspond to a cycle in the above diagram. Similarly, the second permutation has cycle notation $(14625837)$.
%			}
%				\label{figure:cycle-notation}\end{figure}
%		\item Show that every element of the symmetric group has a well defined cycle type.
%		\item Given an element written in cycle notation, define its length to be $n - \text{(the number of cycles)}$. So, for example,
%			$(146837)(2)(5)$ has length $8-3 = 5$ while $(14625837)$ has length $8-1 = 7$.
%			Show that elements of the symmetric group can all be written as products of {\bf transpositions}, where a transposition is a bijection swithing two numbers
%	and fixing the rest. 
%		\item Show that while an element $g \in S_n$ may be written in many different ways as a product of transpositions,
%\todo{aaron: I abandoned this exercise because it was too hard}
%	\end{enumerate}
%	\hint{}
%\end{exercise}




\newpage
\section{The Galois Correspondence}
	
	
	
	
	
	
	
	\subsection{Symmetric Polynomials}	
	
	Let $ P(x) = x^n + a_{n-1} x^{n-1} + \dots + a_1 x + a_0$ be a polynomial with roots $ r_1, r_2, \dots, r_n$.
	
	\begin{exercise}
		Express the coefficients $ a_i$ of $ P(x)$ in terms of $ r_i$.
	\end{exercise}
	Note that each $ a_i$ is a (multi-variable) polynomial in the roots $ r_i$. These polynomials are called the \textbf{elementary symmetric polynomials}.
	The symmetric group $ S_n$ naturally \textbf{acts} on the set of polynomials in $ n$ variables by permuting the variables. 
	That is, we can ``apply'' $\sigma$ to a polynomial $Q$ to obtain the polynomial $\sigma(Q)$ defined by
		\begin{align*}
			(\sigma(Q))(r_1, r_2, \dots, r_n) := Q(r_{\sigma(1)}, r_{\sigma(2)}, \dots, r_{\sigma(n)})
		\end{align*}
		\begin{definition}
			\label{definition:}
			A polynomial $Q(r_1, r_2, \dots, r_n)$ is {\bf symmetric} if for all $\sigma \in S_n $, 
	\begin{align*}
		\sigma(Q) (r_1, r_2, \dots, r_n)= Q(r_1, r_2, \dots, r_n)
	\end{align*}
	In other words, the polynomial remains the same after reordering the variables.
		\end{definition}
		
	\begin{exercise}
		Convince yourself that the elementary symmetric polynomials are symmetric.
	\end{exercise}
	
	Hence the `symmetric' in elementary symmetric. The reason for the `elementary' is the following theorem. 
	
	\begin{theorem}
		\label{theorem:fundamental_theorem_symmetric_polynomials}
		Any symmetric polynomial can be expressed as a polynomial in the elementary symmetric ones. 
		Hence, any symmetric polynomial $ Q(r_1, r_2, \dots, r_n)$ in the roots $ r_i$ of a polynomial $ P(x)$ can be expressed as a polynomial in it's coefficients $ a_i$.
	\end{theorem}
	\begin{remark}
		You can assume this theorem without proof. The proof of this theorem, which is just a careful application of the multinomial theorem and some multi-index induction, gives an algorithm for finding what these polynomials are.
	\end{remark}
	\begin{exercise}
		As an example, show that $ r_1^2 + r_2^2 + \dots + r_n^2  = a_{n-1}^2 - 2 a_{n-2}$.
	\end{exercise}
	
	We've already encountered several examples of this:
	\begin{exercise}[\textbf{Discriminant}]
		Verify that for any polynomial $ P(x)$ the discriminant $\Delta(P) = \prod_{1 \leq i < j \leq n} (r_i - r_j)^2$ is symmetric, and hence by \autoref{theorem:fundamental_theorem_symmetric_polynomials} can be expressed as in terms of the coefficients of $ P(x)$! (which \emph{justifies} the existence of the formulae $ b^2 - 4c, -4p^3 - 27q^2$.)
	\end{exercise}
	
	\begin{exercise}[\textbf{Cubic}] Recall that if $ P(x)$ is a cubic then the intermediate variables \begin{align*}
    \mu_1 &= r_1 + r_2 \omega + r_3 \omega^2 \\
    \mu_2 &= r_1 + r_2 \omega^2 + r_3 \omega
  \end{align*}
		\begin{enumerate}
			\item Show that $ \mu_1^3 + \mu_2^3 $ and $ \mu_1^3  \mu_2^3$ are symmetric. 
			\item Also show that $ \mu_1 + \mu_2$ is \emph{not} symmetric. This was the reason we could not use $ \mu_1, \mu_2$ as our intermediate variables.
		\end{enumerate}
	\end{exercise}
	
	\begin{exercise}[\textbf{Quartic}]
		Recall that if $ P(x)$ is a quartic then the intermediate variables
		\begin{align*}
				\lambda_1 &= r_1 r_2 + r_3 r_4 \\
				\lambda_2 &= r_1 r_3 + r_2 r_4 \\
				\lambda_3 &= r_2 r_3 + r_1 r_4 
		\end{align*}
			\begin{enumerate}
			\item Show that $ \lambda_1 + \lambda_2 + \lambda_3$, $ \lambda_1 \lambda_2 + \lambda_1 \lambda_3 + \lambda_2 \lambda_3$, and $ \lambda_1 \lambda_2 \lambda_3$ are symmetric in $ r_i$. 
			\item Conclude that the coefficients of $ (x-\lambda_1)(x-\lambda_2)(x-\lambda_3)$ can be expressed in terms of the coefficients of $ P(x)$.
		\end{enumerate}
	\end{exercise}
	
	Hopefully now you're convinced that there was some reason behind choosing these intermediate variables. Which brings us to the question: Do such variables exist for all degrees?
	
	
	
	
	
	
	\subsection{Fixed points}
		A set of polynomials $ \mathscr{S} = \{ Q_i(r_1, r_2, \dots, r_n) \}_i$ is said to be \textbf{fixed point-wise} by an element $ \sigma \in S_n$ if 
			\begin{align*}
			(\sigma(Q_i))(r_1, r_2, \dots, r_n) = Q_i(r_{\sigma(1)}, r_{\sigma(2)}, \dots, r_{\sigma(n)})
			\end{align*}
		for every $ Q_i \in \mathscr{S}$. 
		
		\begin{remark}
			The symmetric polynomials are fixed point-wise by every element of $ S_n$. 
		\end{remark}
		
		\begin{remark}
			For a set of polynomials $ \mathscr{S}$, the subset of elements of $ S_n$ that fix $ \mathscr{S}$ point-wise is always a subgroup of $ S_n$, called the \textbf{stabilizer} of $ \mathscr{S}$.
		\end{remark}
		
		\begin{exercise} Let $ P(x)$ be a cubic polynomial. 
			\begin{enumerate}

				\item Find $ \omega \cdot \mu_1$ and $ \omega^2\cdot \mu_1$ in terms of the roots $ r_1, r_2, r_3$.
				\item Find the subgroup of $ S_3$ which fixes the set $\{ \mu_1^3, \mu_2^3 \}$.
				 What do the elements not in this subgroup do to it? (Remember that $ \omega^3 = 1$.)
				\item Find the subgroup of $ S_3$ which fixes the set $\{ \mu_1, \mu_2 \}$ point-wise.
			\end{enumerate}
		\end{exercise}
		
		\begin{exercise}
			Let $ P(x)$ be a quartic polynomial. Find the subgroup of $ S_4$ which fixes the set $ \{ \lambda_1,  \lambda_2, \lambda_3 \}$ point-wise. Have you seen this subgroup before? What do the elements not in this subgroup do to it?
		\end{exercise}
			
			
			
		
		
		
		
		
		\subsection{The Correspondence}
		\todo{Apurva: Please add/ modify/ edit/ expand this. This needs to be very clearly communicated, I'm not sure I'm doing it right.}
		Let us summarize what we have so far:
		
			\textbf{Cubic:} The intermediate variables $ \{\mu_1^3, \mu_2^3 \}$ are fixed by the subgroup $A_3$ which is the commutator $ [S_3, S_3]$, and the commutator $ [A_3, A_3]$ is trivial.
			
			\textbf{Quartic:} The intermediate variables $ \{\lambda_1, \lambda_2, \lambda_3 \}$ are fixed by the subgroup $K_4$ which is the commutator $ [A_4, A_4]$ which in turn is the commutator $ [S_4, S_4]$, and the commutator $ [K_4, K_4]$ is trivial.\\
			
			Galois' idea is the following:
			\begin{enumerate}
				\item If we can solve a general degree $ n$ polynomial by radicals, then we should be able to find a sequence of intermediate variables which satisfy lower degree polynomials. 
				\emph{This part is not hard if you think about what a solution in terms of radicals means.}
				\item Such intermediate variables should be fixed by subgroups of $ S_n$ which can be obtained by successively taking commutators of $ S_n$. \emph{This is Galois' great discovery.} [For those who of you who know what Galois theory is about, this is the Galois correspondence for the field extension of the function fields $ \mathbb{C}(a_0,a_1,\dots, a_{n-1}) \subseteq \mathbb{C}(r_0,r_1,\dots, r_{n-1})$ where the $ r_i$ are free variables and the $ a_i$ are the elementary symmetric polynomials in the $ r_i$.]
				\item For $ n \ge 5$ no such subgroups exist because for $ n \ge 5$ the commutators stabilize: $[S_n, S_n] = A_n$ and $ [A_n, A_n] = A_n$. \emph{This is proven by explicit computation of the commutators.}
			\end{enumerate}
      
      The sequence of successive commutators for the Symmetric groups looks as such
      \begin{align*}
        \mathbf{S_3}: & S_3 \supset [S_3, S_3] = A_3 \supset [A_3, A_3] = \{ e \} \\
				\mathbf{S_4}: & S_4 \supset [S_4, S_4] = A_4 \supset [A_4, A_4] = K_4 \supset [K_4, K_4] = \{ e \} \\
				\mathbf{S_5}: & S_5 \supset [S_5, S_5] = A_5 \supset [A_5, A_5] = A_5
      \end{align*}
      
      
		
			Combine all this and you get
			\begin{theorem}
				A general polynomial of degree $ \ge 5$ cannot be solved using radicals.
			\end{theorem}
		
			




\newpage
\theendnotes
\end{document}


