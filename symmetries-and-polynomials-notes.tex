%        File: symmetries-and-polynomials-notes.tex
%     Created: Thu May 17 10:00 AM 2018 P
% Last Change: Thu May 17 10:00 AM 2018 P
%
%%%%%%%%%%%%%%%%%%%%%
%   AMS packages    %
%%%%%%%%%%%%%%%%%%%%%
\documentclass[12 pt]{amsart}

\usepackage{hyperref}
\usepackage{etex}
\usepackage[shortlabels]{enumitem}
\usepackage{amsmath}
\usepackage{amsxtra}
\usepackage{amscd}
\usepackage{amsthm}
\usepackage{amsfonts}
\usepackage{amssymb}
\usepackage{eucal}
\usepackage[all]{xy}
\usepackage{graphicx}
\usepackage{tikz-cd}
\usepackage{mathrsfs}
\usepackage{subfiles}
\usepackage{mathpazo}
%\usepackage{euler}
\usepackage[colorinlistoftodos, textsize=tiny]{todonotes}
\usepackage{morefloats}
\usepackage{pdfpages}
\usepackage{thm-restate}
\usepackage[utf8]{inputenc}
\usepackage{epigraph}
\usepackage{csquotes}
\usepackage[margin=1.5in]{geometry}

\graphicspath{ {images/} }

\RequirePackage{color}
\definecolor{myred}{rgb}{0.75,0,0}
\definecolor{mygreen}{rgb}{0,0.5,0}
\definecolor{myblue}{rgb}{0,0,0.65}

\usepackage{hyperref}
\hypersetup{citecolor=blue}
\usepackage{tikz}
\usetikzlibrary{matrix,arrows,decorations.pathmorphing}

\theoremstyle{plain}
\newtheorem{theorem}{Theorem}[section]
\newtheorem{proposition}[theorem]{Proposition}
\newtheorem{lemma}[theorem]{Lemma}
\newtheorem{corollary}[theorem]{Corollary}
\theoremstyle{definition}
\newtheorem{definition}[theorem]{Definition}
\newtheorem{remark}[theorem]{Remark}
\newtheorem{example}[theorem]{Example}
\newtheorem{exercise}[theorem]{Exercise}
\newtheorem{counterexample}[theorem]{Counterexample}
\newtheorem{convention}[theorem]{Convention}
\newtheorem{question}[theorem]{Question}
\newtheorem{conjecture}[theorem]{Conjecture} 
\newtheorem{goal}[theorem]{Goal}
\newtheorem{warn}[theorem]{Warning}
\newtheorem{fact}[theorem]{Fact}
\theoremstyle{remark}
\newtheorem{notation}[theorem]{Notation}
\numberwithin{equation}{section}
  
\newcommand\nc{\newcommand}
\nc\on{\operatorname}
\nc\renc{\renewcommand}
\newcommand\se{\section}
\newcommand\ssec{\subsection}
\newcommand\sssec{\subsubsection}
\newcommand\bn{{\mathbb N}}
\newcommand\bc{{\mathbb C}}
\newcommand\br{{\mathbb R}}
\newcommand\bq{{\mathbb Q}}
\newcommand\bp{{\mathbb P}}
\newcommand\CF{{\mathcal F}}
\newcommand\bz{{\mathbb Z}}
\newcommand\ba{{\mathbb A}}
\newcommand\bg{{\mathbb G}}
\newcommand\fa{{\mathfrak a}}
\newcommand\fp{{\mathfrak p}}
\newcommand\fq{{\mathfrak q}}
\newcommand\fm{{\mathfrak m}}
\newcommand\fg{{\mathfrak g}}
\newcommand\fh{{\mathfrak h}}
\newcommand\fgl{{\mathfrak {gl}}}
\newcommand\fsu{{\mathfrak {su}}}
\newcommand\fsl{{\mathfrak {sl}}}
\newcommand\fso{{\mathfrak {so}}}
\newcommand\fo{{\mathfrak {o}}}
\newcommand\fu{{\mathfrak {u}}}
\newcommand\fsp{{\mathfrak {sp}}}
\newcommand\fgsp{{\mathfrak {gsp}}}



\newcommand\sca{\mathscr A}
\newcommand\scb{\mathscr B}
\newcommand\scc{\mathscr C}
\newcommand\scd{\mathscr D}
\newcommand\sce{\mathscr E}
\newcommand\scf{\mathscr F}
\newcommand\scg{\mathscr G}
\newcommand\sch{\mathscr H}
\newcommand\sci{\mathscr I}
\newcommand\scj{\mathscr J}
\newcommand\sck{\mathscr K}
\newcommand\scl{\mathscr L}
\newcommand\scm{\mathscr M}
\newcommand\scn{\mathscr N}
\newcommand\sco{\mathscr O}
\newcommand\scp{\mathscr P}
\newcommand\scq{\mathscr Q}
\newcommand\scs{\mathscr S}
\newcommand\sct{\mathscr T}
\newcommand\scu{\mathscr U}
\newcommand\scv{\mathscr V}
\newcommand\scw{\mathscr W}
\newcommand\scx{\mathscr X}
\newcommand\scy{\mathscr Y}
\newcommand\scz{\mathscr Z}

\newcommand \ra{\rightarrow}
\newcommand \xra{\xrightarrow}
\DeclareMathOperator\spec{\text{Spec }}
\DeclareMathOperator\proj{\text{Proj }}
\DeclareMathOperator\rspec{\textit{Spec }}
\DeclareMathOperator\rproj{\textit{Proj }}
\newcommand*{\shom}{\mathscr{H}\kern -.5pt om}
\newcommand*{\stor}{\mathscr{T}\kern -.5pt or}
\newcommand*{\sext}{\mathscr{E}\kern -.5pt xt}
\newcommand \mg{{\mathscr M_g}}

\makeatletter
\newcommand{\customlabel}[2]{\protected@write \@auxout {}{\string \newlabel {#1}{{#2}{\thepage}{#2}{#1}{}} }\hypertarget{#1}{#2}}
\newcommand\sh[1]{\sco_{#1}^{\rm{sh}}}
\DeclareMathOperator\id{id}
\DeclareMathOperator\tor{Tor}
\renewcommand\hom{Hom}
\DeclareMathOperator\coker{coker}
\DeclareMathOperator\ord{ord}
\DeclareMathOperator\hilb{Hilb}
\DeclareMathOperator\rk{rk}
\DeclareMathOperator\di{div}
\DeclareMathOperator\pic{Pic}
\DeclareMathOperator\lcm{lcm}
\DeclareMathOperator\rank{rank}
\DeclareMathOperator\codim{codim}
\DeclareMathOperator\vol{Vol}
\DeclareMathOperator\supp{Supp}
\DeclareMathOperator\spn{Span}
\DeclareMathOperator\im{im}
\DeclareMathOperator\End{End}
\DeclareMathOperator\sym{Sym}
\DeclareMathOperator\pgl{PGL}
\DeclareMathOperator\sat{Sat}
\DeclareMathOperator\blow{Bl}
\renewcommand\sp{\mathrm{Sp}}
\DeclareMathOperator\gsp{GSp}
\DeclareMathOperator\sgn{sgn}
\DeclareMathOperator\gal{gal}
\DeclareMathOperator\tr{tr}
\renewcommand\char{char}
\newcommand\bbf{{\mathbb F}}
\newcommand\bk{{\Bbbk}}
\newcommand\ul{\underline}
\newcommand\ol{\overline}
\DeclareMathOperator\pr{pr}
\DeclareMathOperator\ev{ev}
\DeclareMathOperator\maj{maj}
\DeclareMathOperator\inv{inv}
\DeclareMathOperator\isom{isom}
\DeclareMathOperator\mor{mor}
\DeclareMathOperator\aut{Aut}
\DeclareMathOperator\gl{GL}
\renewcommand\sl{\mathrm{SL}}
\DeclareMathOperator\mat{Mat}
\DeclareMathOperator\stab{Stab}
\DeclareMathOperator\so{SO}
\DeclareMathOperator\su{SU}
\renewcommand\u{\mathrm{U}}
\DeclareMathOperator\lie{Lie}
\DeclareMathOperator\ad{ad}
\DeclareMathOperator\Ad{Ad}
\renewcommand\o{{\rm{O}}}




\setcounter{MaxMatrixCols}{20}

\def\listtodoname{List of Todos}
\def\listoftodos{\@starttoc{tdo}\listtodoname}

\title{Symmetries and Polynomials Notes}
\author{Aaron Landesman and Apurva Nakade}

\usepackage{microtype}
\begin{document}

\maketitle

\section{The Discriminant}

\subsection{Quadratic Polynomials}
\begin{definition}[Quadratic discriminant]
	\label{definition:quadratic-discriminant}
	Let $p(x) = x^2 + bx + c$ be a polynomial with $b$ and $c$ real numbers.
	The discriminant $\Delta(p)$ is by definition $b^2 - 4c$.
\end{definition}
\begin{exercise}
	\label{exercise:}
	If the polynomial $p(x) = x^2 + bx + c$ has roots $\alpha$ and $\beta$, express $b$ and $c$
	in terms of $\alpha$ and $\beta$.
\end{exercise}
\begin{exercise}
	\label{exercise:quadratic-roots}
	What does the sign of the discriminant (i.e., whether $\Delta(p) > 0, < 0$ or $=0$) 
	tell you about the roots $\alpha$ and $\beta$?
	{\it Hint:} If the discriminant is $>0$ show that the roots are real, if it is equal
	to $0$, show they are the same, if it is less than $0$, show they are complex numbers
	which are not real.
\end{exercise}
\begin{exercise}
	\label{exercise:quadratic-discriminant-in-roots}
	Express the discriminant of the polynomial $p(x) = x^2 + bx + c$ in terms of
	the roots $\alpha, \beta$.
\end{exercise}

\subsection{The discriminant in general}
\begin{definition}
	\label{definition:discriminant}
	For $p(x)$ a polynomial of the form $p(x) = (x-r_1)(x-r_2) \cdots (x-r_n)$,
	define the discriminant $\Delta(p) := \prod_{1 \leq i < j \leq n} (r_i - r_j)^2$.
\end{definition}
\begin{exercise}
	\label{exercise:}
	Verify that for $p(x)$ of degree $2$, the definition of the discriminant
	of a general polynomial from \autoref{definition:discriminant} agrees with
	that of a quadratic polynomial given in \autoref{definition:quadratic-discriminant}.
	{\it Hint:} Use \autoref{exercise:quadratic-discriminant-in-roots}.
\end{exercise}
The following exercise shows that the discriminant measures, in a precise sense, how far apart
the roots of a polynomial are.
\begin{exercise}
	\label{exercise:}
	Show that for $p$ a polynomial, $\Delta(p) = 0$ if and only if $p$ has a repeated root.
\end{exercise}
\subsection{Cubic discriminants}
\begin{exercise}
	\label{exercise:real-root-cubic}
	Show that a cubic polynomial with real coefficients $p(x) = x^3 + ax + bx + c$
	always has a real root.
	{\it Hint:} Graph the cubic and show it intersects the line $p(x) = 0$.
\end{exercise}
\begin{exercise}
	\label{exercise:}
	Using \autoref{exercise:real-root-cubic}, show that a cubic polynomial either has
	$3$ real roots or $2$ complex conjugates roots (of the form $a + bi, a-bi$ for $a,b$ real numbers) and one real root. 
	{\it Hint:} Factor out the real root, and use your understanding of quadratic polynomials.
\end{exercise}
\begin{exercise}
	\label{exercise:discriminant-cubic-sign}
	For $p(x) = x^3 + ax^2 + bx + c$ a cubic polynomial with real coefficients, show $\Delta(p) = 0$ if and only 
	if there is a repeated root, $\Delta(p) > 0$ if and only if $p$ has three distinct real roots, and $\Delta(p) < 0$ if and only if $p$ has two complex conjugate roots and one real root.
	Compare your answer to \autoref{exercise:quadratic-roots}.
\end{exercise}

\todo{aaron: i didn't discuss the formula for the cubic discriminant, but i think this is already perhaps too much material for one day, i think maybe you can just state it as a fact the formula that the discriminant of $x^3 + ax + b$ is $-4a^3-27b^2$ or whatever it is (and optionally allow them to expand it out but suggest they don't prove it since it seems boring and tedious) }

{\it Your homework is to complete up through \autoref{exercise:discriminant-cubic-sign}. If you finish that, and still have time try the following challenge questions.}

\subsection{Counting polynomials of discriminant $0$}
\begin{exercise}[Challenge 1]
	\label{exercise:repeated-roots}
	Let $p(x) = x^n + a_{n-1}x^{n-1} + \cdots + a_0$, where now the coefficients $a_i$ are
	in $\bz/p$ (i.e., take on values between $0$ and $p-1$), and the discriminant is
	also considered as a number in $\bz/p$. Show there are $p^n$ such polynomials, and exactly
	$p^{n-1}$ of them have discriminant $0$.
	Conclude that the number of squarefree polynomials of degree $n$ over $\bz/p$ is $p^n - p^{n-1}$.
	{\it Hint:} Try factoring the polynomials over $\bz/p$, and write each polynomial
	uniquely as $f(x)g(x)^2$, where $f(x)$ is squarefree. Then count the number
	of such polynomials for the degree of $f$ fixed inductively.	
\end{exercise}

\begin{exercise}[Challenge 2]
	\label{exercise:common-roots}
	Using a similar method to that of \autoref{exercise:repeated-roots},
	count the number of pairs of degree $n$ polynomials $(p,q)$ for 
	$p(x) = x^n + a_{n-1}x^{n-1} + \cdots + a_0$ and $q(x) = x^n + b_{n-1}x^{n-1} + \cdots + b_0$
	with $a_i \in \bz/p, b_i \in \bz_p$ so that $p$ and $q$ have no common root $\bmod p$.
\end{exercise}
\begin{remark}
	\label{remark:}
	If you did the two prior exercises \autoref{exercise:repeated-roots} and \autoref{exercise:common-roots} correctly, you may notice a striking similarity
between the two answers. There is in deed a deeper connection, but the answer lies deep.
Loosely speaking, if you take a polynomial $p(x)$ with no repeated roots, you can send it
to the pair of polynomials $(p(x) + p'(x), p(x))$. Here $p'(x)$ denotes the derivative of $p(x)$.
If $p(x) = \sum_i a_ix^i$ then $p'(x) = \sum_i  i \cdot a_i x^{i-1}$.
The diligent reader will check that this is a map from the space of polynomials with no repeated roots to the
space of pairs of polynomials with no repeated roots.
In some sense (which we do not explain) this map explains why the counts from 
\autoref{exercise:repeated-roots} and \autoref{exercise:common-roots} are so similar.
\end{remark}





\newpage
\section{Solving the Cubic}

\newpage
\section{Symmetry Groups}

\newpage
\section{The Galois Correspondence}

\newpage
\section{Solving the Quartic}

\bibliographystyle{alpha}

\end{document}


